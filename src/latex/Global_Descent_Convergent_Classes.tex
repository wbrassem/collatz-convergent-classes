\documentclass[12pt, oneside]{article}   	% use "amsart" instead of "article" for AMSLaTeX format
%\documentclass[12pt, oneside, draft]{article}   	% use "amsart" instead of "article" for AMSLaTeX format

%%%%%%%%%%%%%%%%%%%%%%%%%%%%%%%%%%%%%%%%%%%%%%%%%
%
% The TexLive distribution is stored in /usr/local/texlive/2020
%
% The definitions of the packages are in /usr/local/texlive/2020/texmf-dist/tex/latex
%
%%%%%%%%%%%%%%%%%%%%%%%%%%%%%%%%%%%%%%%%%%%%%%%%%

\usepackage{xparse}					% Allows for the creation of \NewDocumentCommand shortcuts

\usepackage{geometry}				% See geometry.pdf to learn the layout options. There are lots.
\geometry{letterpaper}				% ... or a4paper or a5paper or ... 
\usepackage[parfill]{parskip}		% Activate to begin paragraphs with an empty line rather than indent
\usepackage{float}					% Float package to exert more control over graphic placement
	
%\usepackage{verbatim}				% Creates a verbatim environment for code / program output
\usepackage{graphicx}				% Use pdf, png, jpg, or eps§ with pdflatex; use eps in DVI mode
									% TeX will automatically convert eps --> pdf in pdflatex
\graphicspath{{./figures/}}         

% Math packages
\usepackage{amssymb}				% Mathematical symbols
\usepackage{amsmath}				% Added for typesetting mathematical formulae
\usepackage{amsthm}					% Added for typesetting mathematical theorems
\usepackage{mathalpha}				% Added for some special characters like Q, Z and N

% Table packages in addition to built in tabular
\usepackage{booktabs}				% Adds extra commands to tabular like \toprule
\usepackage{afterpage}				% For pagination support
\usepackage{longtable}				% Support for longer tables

% Hyperlinks
\usepackage{hyperref}				% For creating internal and HTTP links

% References
\usepackage[numbers]{natbib}        % For bibliography management

% For Tikz graphics
\usepackage{tikz}					% Package for pgf and TikZ
\usetikzlibrary{shapes,arrows,chains}
\tikzstyle{startstop} = [very thick, rectangle, rounded corners, minimum width=2.5cm, minimum height=0.8cm, text centered, draw=black, text width=2.5cm]
\tikzstyle{action} = [rectangle, rounded corners, minimum width=3cm, minimum height=0.8cm, text centered, draw=black, text width=2.5cm]
\tikzstyle{decision} = [diamond, minimum width=3cm, minimum height=3cm, text centered, draw=black, text width=2cm]
\tikzstyle{arrow} = [thick,->,>=stealth]

% For theorems and definitions - each creates it's own counter which incrementa each use
\theoremstyle{definition}
\newtheorem{proposition}{Proposition}[section]
\newtheorem{define}{Definition}[section]
\newtheorem{axiom}{Axiom}[section]
\newtheorem{theorem}{Theorem}
\newtheorem{lemma}[theorem]{Lemma}
\newtheorem{corollary}[theorem]{Corollary}
\newtheorem*{remark}{Remark}

%SetFonts


\title{Global Descent and Convergent Classes in the Collatz Conjecture}
\author{Wayne Brassem}
%\date{}							% Activate to display a given date or no date


\begin{document}

% Document define commands for commonly used items in math mode
\NewDocumentCommand{\setN}{}{\mathbb{N}}				% Set of positive integers, excluding 0
\NewDocumentCommand{\setNo}{}{\mathbb{N}_0}				% Set of positive integers, including 0
\NewDocumentCommand{\setNeven}{}{\mathbb{N}_{even}}		% Set of even positive integers, excluding 0
\NewDocumentCommand{\setNodd}{}{\mathbb{N}_{odd}}		% Set of odd positive integers
\NewDocumentCommand{\setZ}{}{\mathbb{Z}}				% Set of all integers, excluding zero
\NewDocumentCommand{\setZo}{}{\mathbb{Z}_0}				% Set of all integers, including zero
\NewDocumentCommand{\setQ}{}{\mathbb{Q}}				% Set of all rational numbers

% Document define commands for commonly used items not in math mode
\NewDocumentCommand{\Rarr}{}{\textrightarrow{}}			% Right arrow

% Document commands which provide hyperlinks to OEIS sequences referenced herein
\NewDocumentCommand{\OEIS}{m}{\href{https://oeis.org/#1}{#1}}

\maketitle



\begin{abstract}

We study the accelerated Collatz map
\[
f(x) = \begin{cases}
\displaystyle
    \phantom{3x} \frac{x}{2} & \text {if } x \text{ is even}, \\[8pt]  % Insert some spacing between rows
\displaystyle
    \frac{3x+1}{2}           & \text {if } x \text{ is odd}. \\
\end{cases}
\]
and analyze integer trajectories through parity-encoded residue classes.
Each finite parity word determines a non-overlapping residue class modulo
a power of two, and convergent parity words correspond to classes whose
elements reach a strictly smaller integer after finitely many iterations.

Using explicit enumeration of admissible parity words and their associated
division counts, we show that each convergent residue class admits a
well-defined asymptotic density. Grouping these classes by minimal stopping
time yields a disjoint family whose cumulative density increases
monotonically. We prove that the density contribution of newly appearing
convergent classes tends to zero as the stopping time grows. Since these
classes exhaust all non-convergent residue classes, the complement of the
convergent union has asymptotic density zero.

Consequently, the cumulative density of convergent residue classes converges
to one. Every positive integer therefore belongs to a convergent residue
class of finite stopping time and eventually reaches a smaller positive
integer under iteration of the Collatz map. This establishes global descent
and rules out the existence of nontrivial cycles in the positive integers,
leaving the classical \(4 \rightarrow 2 \rightarrow 1 \rightarrow 4\) cycle
as the unique periodic orbit.

\end{abstract}


\tableofcontents
\newpage


\section{Introduction}

\subsection{The Collatz Mapping and Known Results}

The Collatz conjecture concerns the behavior of the accelerated map defined below
on the positive integers.

\begin{define}
\label{def:accelerated-map}
The \emph{accelerated Collatz map} is the function
\( f : \setN \rightarrow \setN \) defined by
\[
f(x) = \begin{cases}
\displaystyle
    \phantom{3x} \frac{x}{2} & \text {if } x \text{ is even}, \\[8pt]
\displaystyle
    \frac{3x+1}{2}           & \text {if } x \text{ is odd}. \\
\end{cases}
\]
This map acts on the positive integers and combines each odd step of the classical Collatz
iteration \(x \mapsto 3x+1\) with the immediately following division by two
\cite{Everett1977}.
\end{define}

The Collatz conjecture asserts that, under iteration of \(f\), every positive integer
eventually reaches the cycle \(4 \to 2 \to 1 \to 4\) \cite{Collatz1937}.

Despite its elementary formulation, the conjecture has resisted proof for decades
\cite{Lagarias2010,Terras1976}. Extensive computational verification confirms convergence
for all integers up to very large bounds \cite{OliveiraESilva2010}, but such verification
does not constitute a proof and offers limited insight into the global structure
of Collatz trajectories.

Many prior approaches focus on bounding individual trajectories or probabilistic heuristics 
\cite{Wirsching1998}. While these methods provide valuable intuition, they do not address
the problem at the level of the entire positive integer space.

\subsection{Strategy of the Proof}

This paper adopts a structural approach based on decomposing the
positive integers into arithmetic residue classes determined by parity
patterns under the accelerated Collatz map.

The central idea is to study finite Collatz paths abstractly, independent
of their starting values, and to associate each such path with a
pairwise disjoint residue class of integers that follow the same parity
pattern for a fixed number of odd steps. Each class is assigned a
minimal stopping time, defined as the first iterate at which its
elements strictly decrease.

The proof proceeds through the following steps:
\begin{itemize}
\item Encode Collatz trajectories using parity words that record the
      number of divisions by two between odd steps.
\item Show that each admissible parity word induces an arithmetic
      residue class of positive integers.
\item Prove that convergent residue classes of fixed stopping time are
      pairwise disjoint.
\item Compute the asymptotic density of each class using its associated
      total division count.
\item Show that the density contribution of newly appearing convergent
      classes tends to zero as the stopping time increases, and that
      these classes exhaust all non-convergent residue classes.
\item Conclude that every positive integer eventually reaches a smaller
      value, establishing global descent and ruling out non-trivial
      cycles.
\end{itemize}

By focusing on exhaustion of the positive integers via convergent
structures rather than on individual trajectories, this approach
provides a global and deterministic mechanism for descent under
the Collatz map.


\section{Definitions and Preliminaries}

\subsection{Stopping Time}

\begin{define}
The stopping time \( \sigma(x) \) of a positive integer \(x\) is the minimal integer
\(k \in \setN\) such that
\[
f^{k}(x) < x.
\]
Stopping time has been studied in the context of the $3x+1$ problem
\cite{Terras1976}.
\end{define}

\subsection{Convergent Subsequences}

\begin{define}
A \emph{convergent subsequence of length \(k\)} is a finite sequence
\[
x, f(x), f^2(x), \ldots, f^k(x)
\]
such that \( f^k(x) < x \) and \(k\) is minimal. Finite subsequences of this type were
considered in the context of the accelerated Collatz map by Everett \cite{Everett1977}.
\end{define}

\subsection{Residue Classes}

\begin{define}
Two positive integers \(x,y \in \setN\) are said to be in the same
\emph{convergent residue class of stopping time \(k\)} if their first
\(k\) Collatz iterates are identical. The idea of encoding trajectories via residue classes
or arithmetic progressions has appeared in prior works on symbolic and structural approaches
to the $3x+1$ problem \cite{Wirsching1998,Chamberland2003}.
\end{define}

\begin{define}[Convergent Residue Class]
Fix \(k \in \setN\) and \(x \in \setN\). The \emph{convergent residue class of stopping time \(k\)}
containing \(x\) is
\[
[x]_k := \{ y \in \setN \mid f^i(y) = f^i(x) \text{ for } i = 0, 1, \dots, k-1 \},
\]
where \(f\) is the accelerated Collatz map.  \([x]_k\) collects all positive integers whose
first \(k\) Collatz iterates are identical to those of \(x\).
\end{define}


\section{Division Counts and Path Encoding}

The structure of convergent residue classes is governed by the total number
of divisions by two incurred along a Collatz path, which depends only on the
parity pattern of the path and not on the specific starting integer.

\subsection{Total Division Count}

Let a Collatz path contain exactly \(k\) odd steps (``up-legs''). Between
successive odd steps, the accelerated Collatz map applies a finite number
of divisions by two. The cumulative number of such divisions determines
the arithmetic structure of the corresponding residue class.

\begin{define}[Total Division Count]
Let \(k \ge 1\). Define \(A020914(k)\) to be the total number of divisions
by two applied along an accelerated Collatz path with exactly \(k\)
odd steps.
\end{define}

Although admissible parity words with the same number of odd steps may differ
in the distribution of divisions by two between steps, the total number of
divisions is invariant. For any Collatz path with exactly \(k\) odd steps,
the sum of the division counts equals \(A020914(k)\). This quantity depends
only on the number of odd steps and not on the specific parity word.

This quantity coincides with the number of binary digits of \(3^k\),
as tabulated by \OEIS{A020914}, but its role here is purely arithmetic:
it records the total exponent of 2 accumulated in the denominator of the
iterated map, a perspective implicit in earlier analyses of stopping times
and parity sequences \cite{Terras1976,Wirsching1998}.

\subsection{Residue Class Modulus}

The correspondence between fixed parity patterns and arithmetic residue
classes modulo powers of two is well known in the study of Collatz dynamics
\cite{Wirsching1998,Chamberland2003}.

Fix a parity pattern with exactly \(k\) odd steps and total division count
\(A020914(k)\). All integers realizing this parity pattern form an
arithmetic progression with common difference \(2^{A020914(k)}\).

\begin{lemma}
Let \(x,y \in \setN\) follow the same Collatz parity pattern of length \(k\).
Then
\[
x \equiv y \pmod{2^{A020914(k)}}.
\]
\end{lemma}

\begin{proof}
Each odd step contributes a factor of 3 in the numerator, while each even
step contributes a factor of 2 in the denominator. After \(k\) odd steps,
the accumulated denominator is exactly \(2^{A020914(k)}\), independent of
the starting integer, which fixes the residue class modulo this power of two.
\end{proof}

\subsection{Density of Residue Classes}

\begin{proposition}
Each residue class with \(k\) odd steps has asymptotic density
\[ 2^{-A020914(k)}. \]
\end{proposition}

\begin{proof}
Each class is an arithmetic progression with step size \(2^{A020914(k)}\)
and exactly one representative per period. The result follows.
\end{proof}

\begin{remark}
At fixed \(k\), there are finitely many distinct convergent residue classes.
Let \(\widetilde A(k)\) denote the number of such classes. Since these classes
are pairwise disjoint, their total density contribution at stopping time \(k\) is
\[
\widetilde A(k)\,2^{-A020914(k)}.
\]
\end{remark}

\subsection{Example}

For convergent subsequences with \(k=4\) odd steps, \(A020914(4)=7\).
Thus all associated residue classes consist of integers of the form
\[
x = 2^7 n + r, \quad n \in \setN,
\]
where the residue \(r\) depends on the specific parity pattern.
The spacing between consecutive elements is therefore \(2^7 = 128\).

The preceding example illustrates that convergent behavior depends only
on parity structure and accumulated division count, motivating an abstract
description of Collatz trajectories independent of their starting values.


\section{Grammar of Collatz Parity Words}

\begin{remark}[Parity Grammar as a Dynamical Encoding]
The parity--word grammar introduced in this section provides a symbolic
encoding of accelerated Collatz trajectories. This approach, introduced
by Chamberland~\cite{Chamberland2003}, is not injective: many integers
correspond to the same parity word, forming arithmetic residue classes.

Our use of parity words differs in that we exploit this encoding to
rigorously quantify densities of convergent residue classes and to
establish global descent. While individual trajectories are collapsed
under the encoding, their asymptotic behavior with respect to growth and
contraction is retained. This suffices for density, exhaustion, and
cycle--exclusion arguments.
\end{remark}

\begin{define}[Parity Word]
A \emph{parity word} of length \(k \ge 1\) is a finite sequence
\[
\omega = (d_1, d_2, \dots, d_k),
\]
where each \(d_i \in \setN\) denotes the number of divisions by two
applied after the \(i\)-th odd step of an accelerated Collatz trajectory.
\end{define}

Each parity word determines a linear--affine transformation
\[
x \longmapsto \frac{3^k x + c(\omega)}{2^{d_1+\cdots+d_k}},
\]
where \(c(\omega)\) is a canonical integer uniquely determined by the
parity word (see Appendix~\ref{app:canonical-c}).

\begin{remark}[Parity Words Encode Structure, Not Values]
Fixing a parity word $\omega$ determines the sequence of arithmetic
operations applied along an accelerated Collatz trajectory. Consequently,
both the linear coefficient \(3^k/2^{d_1+\cdots+d_k}\) and the affine term
\(c(\omega)\) depend only on $\omega$ and not on the starting integer.

Thus, convergence or divergence is a property of the parity word itself
and applies uniformly to all sufficiently large elements of the induced
residue class.
\end{remark}

\begin{define}[Admissible Parity Word]
A parity word \(\omega\) is \emph{admissible} if there exists a positive
integer \(x\) realizing \(\omega\) such that the induced affine map
\(T_\omega(x)\) is a positive integer.
\end{define}

\begin{lemma}[Structural Admissibility]
Admissibility of a parity word $\omega$ depends only on the inequalities
satisfied by the tuple $(d_1,\dots,d_k)$ and not on the particular
realizing integer. In particular, admissibility is determined entirely
by whether the affine map \(T_\omega\) maps some integer to
\(\setN\).
\end{lemma}

\begin{lemma}[Minimal Division Count]
For any admissible parity word of length \(k\),
\[
d_1 + \cdots + d_k \ge A020914(k),
\]
where \(A020914(k)\) denotes the minimal total number of divisions by two
required to clear denominators in an accelerated Collatz trajectory with
\(k\) odd steps. Equality holds if and only if the parity word follows the
lexicographically minimal admissible increment pattern (i.e., the parity
word achieving admissibility with minimal total division count).
\end{lemma}

\begin{define}[Convergent Parity Word]
An admissible parity word \(\omega = (d_1,\dots,d_k)\) is
\emph{convergent} if
\[
\frac{3^k}{2^{d_1+\cdots+d_k}} < 1.
\]
\end{define}

\begin{lemma}[Associated Collatz Map]
\label{lem:associated-map}
Each parity word \(\omega = (d_1,\dots,d_k)\) induces an affine map
\[
T_\omega(x) = \frac{3^k x + c(\omega)}{2^{d_1+\cdots+d_k}},
\]
where \(c(\omega)\) is the unique integer satisfying
\[
0 \le c(\omega) < 2^{d_1+\cdots+d_k}
\]
for which \(T_\omega(x)\in\setN\) for all integers realizing \(\omega\).
\end{lemma}

\begin{remark}[Affine Compression of Trajectories]
For any integer realizing a parity word $\omega$, the value
\(T_\omega(x)\) coincides exactly with the result of applying the
accelerated Collatz map for \(k\) successive odd steps. Thus the affine
map \(T_\omega\) represents a compression of the corresponding finite
Collatz trajectory into a single step.
\end{remark}

\begin{lemma}
\label{lem:residue_class}
Each convergent parity word \(\omega\) induces a residue class
\[
x \equiv r_\omega \pmod{2^{A020914(k)}},
\]
whose elements all follow the same Collatz parity pattern for \(k\) odd
steps and reach an integer strictly smaller than the starting value after
at most \(k\) odd steps.
\end{lemma}

\begin{remark}[Increment Grammar from Minimal Denominator Growth]
The first--difference sequence of \(A020914\), given by
\(\OEIS{A022921}\), induces a binary increment grammar
\(\{s,d\}\) describing the growth of the \emph{minimal admissible}
total division count as the number of odd steps increases.

Specifically, each symbol records whether the minimal value of
\(A020914(k)\) increases by one (\(s\)) or by two (\(d\)) when passing from
\(k\) to \(k+1\) odd steps. This increment grammar does \emph{not} encode
the realized parity word entries \(d_i\), which may be arbitrarily large,
but rather governs the evolution of the minimal denominator required for
admissibility.

The horizontal--sum construction used to enumerate convergent parity
words tracks the accumulation of excess divisions above this minimal
baseline, following the algorithmic framework introduced by
Winkler~\cite{Winkler2017}. Consequently, the admissible local patterns
in the \(\{s,d\}\)-grammar determine the combinatorial growth of
\(\widetilde A(k)\) independently of the specific values of the parity
word entries.
\end{remark}

The preceding grammar restricts admissible parity words to a finite,
recursively generated family.
\begin{proposition}
The number of convergent parity words of length \(k\) is given by
\(\OEIS{A186009}\).
\end{proposition}

\begin{lemma}[Single-to-Double Row Sum Bound]
\label{lem:single-double-bound}
Let a row produced by a single step have total sum $k$. Apply a double step
immediately afterward. Then the resulting row has total sum at most $3k$.
\end{lemma}

\begin{proof}
By definition, the total sum of the single row equals $k$. Under a subsequent
double step, the terminal entry of the new row is formed as the sum of all
entries of the preceding row, and the previous terminal entry is duplicated.

Thus the double row consists of:
(i) the interior entries of the single row, whose total is $\le k$;
(ii) one terminal entry equal to $k$; and
(iii) a duplicated terminal entry, also equal to $k$.

Hence the total sum of the double row is bounded by
\[
k + k + k = 3k.
\]
\end{proof}

This lemma ensures that doubling steps contribute at most a controlled
multiplicative factor to horizontal-sum growth, independent of global cycle
structure. Consequently, all admissible local patterns within a $41$-cycle
admit finite maximal growth factors, computed explicitly by finite enumeration
in Appendix~\ref{app:local-growth}.


\section{Cumulative Density of Convergent Classes}

Define
\[
\widetilde A(i) := A186009(i+1),
\]
so that $\widetilde A(i)$ counts convergent residue classes of stopping time $i$.

Let
\[
B(i) := A020914(i),
\]
so that the density contribution of convergent residue classes of stopping time $i$ is
\[
\widetilde A(i)\, 2^{-B(i)}.
\]

By definition of $A020914$ as the minimal exponent with $2^{B(i)} > 3^i$, the function
$B(i)$ satisfies the following properties:
\begin{itemize}
    \item $B(i)$ is strictly increasing, with increments either 1 or 2 and no zeros.
    \item Asymptotically, $B(i) \sim (\log_2 3)\, i$.
    \item By construction, $2^{B(i)} > 3^i$ for all $i \ge 0$.
\end{itemize}

Because $2^{B(i)}$ is a power of two and $B(i)$ is defined as the minimal
integer such that $2^{B(i)} > 3^i$, it follows that $2^{B(i)}$ is the smallest
power of two exceeding $3^i$. Since consecutive powers of two differ by a
factor of $2$, we obtain the sharp bound
\[
3^i < 2^{B(i)} < 2 \cdot 3^i.
\]

\begin{remark}[Indexing and Effective Growth Delay]
The OEIS sequences \(A020914\) and \(A186009\) employ differing indexing
conventions. In order to compare numerator and denominator growth on a common
scale, we define \(\widetilde A(i) := A186009(i+1)\), so that the index \(i\)
corresponds uniformly to stopping time (equivalently, parity-word length).

While the first doubling increment in \(A020914\) occurs at index \(1\), its
effect on \(\widetilde A\) appears after a fixed finite delay of two growth
steps. This delay is independent of \(i\) and therefore does not affect
asymptotic growth-rate comparisons.
\end{remark}

\noindent
\textbf{Constant-factor control.}
This bound depends only on the definition of $A020914$ and does not rely on
any delayed-doubling or numerator-growth arguments. Consequently, the reciprocal satisfies
\[
2^{-B(i)} \in \left( \frac{1}{2 \cdot 3^i}, \frac{1}{3^i} \right),
\]
giving a precise constant factor for the density contribution at level $i$.

We now record this observation formally for later reference.
\begin{lemma}[Asymptotic Control of Density Contributions]
\label{lem:asymptotic-control}
For all $i \ge 0$,
\[
3^i < 2^{A020914(i)} < 2\cdot 3^i.
\]
Consequently, the density contribution of convergent residue classes of
stopping time $i$ satisfies
\[
\frac{\widetilde A(i)}{2 \cdot 3^i}
\;<\;
\widetilde A(i)\,2^{-A020914(i)}
\;<\;
\frac{\widetilde A(i)}{3^i}.
\]
\end{lemma}

\begin{proof}
By definition, $A020914(i)$ is the minimal integer $m$ such that $2^m > 3^i$.
Consecutive powers of two differ by a factor of $2$, so
\[
3^i < 2^{A020914(i)} < 2 \cdot 3^i.
\]
Taking reciprocals and multiplying by $\widetilde A(i)$ gives the stated bounds.
\end{proof}

\begin{theorem}[Maximal Admissible Growth Rate for $\widetilde A = A186009$]
\label{thm:maximal-growth}
Let $\widetilde A(i) := A186009(i+1)$ be defined via the horizontal-sum construction
governed by the increment grammar of \OEIS{A022921}, and let $B(i) := A020914(i)$
denote the cumulative exponent of $2$ up to level $i$.

Then
\[
\limsup_{i\to\infty} \widetilde A(i)^{1/i} < 3,
\]
and consequently
\[
\frac{\widetilde A(i)}{2^{B(i)}} \longrightarrow 0 \quad \text{as } i\to\infty.
\]
\end{theorem}

\begin{proof}
The increment sequence \OEIS{A022921} induces a finite symbolic grammar on symbols
\[
\{ s, d \},
\]
corresponding respectively to single and double events in the horizontal-sum construction.
Every admissible infinite sequence of increments can be partitioned into concatenations
of two fundamental cycle types:
\begin{itemize}
  \item a \emph{7-cycle} with pattern
  \[
  \{ s, d, s, d, s, d, d \},
  \]
  and
  \item a \emph{5-cycle} with pattern
  \[
  \{ s, d, s, d, d \}.
  \]
\end{itemize}

Each cycle contributes a fixed multiplicative factor to the net growth. Explicitly,
\[
\frac{2^{11}}{3^{7}} < 1
\quad\text{for the 7-cycle,}
\qquad
\frac{2^{8}}{3^{5}} > 1
\quad\text{for the 5-cycle.}
\]
Thus 7-cycles are strictly contractive, while 5-cycles are expansive.

The minimal composite block that arises naturally in the grammar is the \emph{12-cycle},
formed by concatenation of a 7-cycle followed by a 5-cycle, with pattern
\[
\{ s, d, s, d, s, d, d, s, d, s, d, d \}.
\]
Its net multiplicative factor is
\[
\frac{2^{19}}{3^{12}} < 1,
\]
so the 12-cycle is also contractive.

Since 12-cycles are contractive, consecutive repetitions multiplicatively diverge from
unity until it becomes admissible to append a 5-cycle. The smallest admissible block
whose net effect exceeds unity is therefore the \emph{41-cycle}, consisting of three
consecutive 12-cycles followed by a 5-cycle. Its net multiplicative factor is
\[
\frac{2^{65}}{3^{41}} > 1.
\]

Any admissible infinite word whose associated horizontal sums grow without bound
is eventually dominated, in per-symbol growth rate, by sequences composed of repeated
41-cycles or by longer blocks containing strictly more contractive content per symbol.
Since the 41-cycle maximizes per-symbol growth among all admissible patterns, 
its infinite repetition provides a computable upper bound on $\widetilde A(i)$,
and therefore bounds the maximal possible geometric growth rate of all admissible sequences.

The horizontal-sum dynamics within each local configuration combine Pascal-type summation
with delayed doubling effects. A complete extremal analysis over all admissible local
patterns occurring within a 41-cycle yields a maximal per-step growth factor
\[
\rho_{41} < 3.
\]
Since all other admissible infinite sequences contain strictly more contractive content,
their geometric growth rates are bounded above by $\rho_{41}$ as well. Therefore,
\[
\limsup_{i\to\infty} \widetilde A(i)^{1/i} \le \rho_{41} < 3.
\]

Finally, since the denominator satisfies
\[
3^i < 2^{B(i)} < 2\cdot 3^i,
\]
we obtain
\[
\frac{\widetilde A(i)}{2^{B(i)}} \le \frac{C\,\rho_{41}^i}{3^i} \longrightarrow 0,
\]
which completes the proof.
\end{proof}


\section{Global Descent}

\begin{theorem}[Global Descent]
Every positive integer eventually reaches a smaller positive integer
under iteration of the Collatz map.
\end{theorem}

\begin{proof}
By Theorem~\ref{thm:maximal-growth}, the density contribution of convergent
residue classes of stopping time $i$ satisfies
\[
\frac{\widetilde A(i)}{2^{B(i)}} \longrightarrow 0 \quad \text{as } i \to \infty.
\]
Consequently, the cumulative density of convergent residue classes approaches 1. 
Since the residue classes form a partition of the positive integers, 
every integer must belong to some convergent residue class of finite stopping time.

For each such residue class, there exists a finite integer $k$ such that
the $k$-fold iteration of the Collatz map satisfies
\[
f^k(x) < x.
\]
Hence every positive integer eventually reaches a smaller positive integer
under iteration of the Collatz map.
\end{proof}

Global Descent implies that no positive integer can remain perpetually above
its initial value, leaving only the trivial 1-2-4 cycle as a recurrent orbit.

\section{Non-Existence of Non-Trivial Cycles}

\begin{theorem}
The only cycle in the positive integers under the Collatz map is
\(4 \rightarrow 2 \rightarrow 1 \rightarrow 4\).
\end{theorem}


\section{Conclusion}

We have shown that the positive integers may be partitioned into
non-overlapping residue classes determined by finite Collatz parity
patterns, each assigned a minimal stopping time. By explicitly
enumerating convergent parity words and controlling the maximal admissible
growth rate of \(\widetilde A(i)\) relative to \(2^{B(i)}\), we proved that
the cumulative density of convergent classes converges to one.

This establishes a global descent mechanism for the accelerated Collatz
map: every positive integer eventually reaches a strictly smaller value,
ruling out the existence of non-trivial cycles. The argument is global
and structural in nature, relying neither on probabilistic heuristics
nor on bounds for individual trajectories.

The approach reframes the Collatz conjecture as an exhaustion problem
over arithmetic classes rather than a question of trajectory control,
suggesting a framework that may be applicable to related discrete
dynamical systems.


\newpage
\appendix
\section{Canonical Determination of the Constant \(c(\omega)\)}
\label{app:canonical-c}

In Lemma~\ref{lem:associated-map}, each parity word
\(\omega = (d_1,\dots,d_k)\) induces an affine Collatz map of the form
\[
T_\omega(x) = \frac{3^k x + c(\omega)}{2^{D}}, 
\qquad D = d_1 + \cdots + d_k.
\]
The constant \(c(\omega)\) is uniquely determined by the parity word
\(\omega\) up to congruence modulo \(2^D\). In this appendix, we describe
a canonical choice of representative and an explicit method for
computing it.

\subsection*{Existence and Uniqueness}

Fix a parity word \(\omega\) of length \(k\) and let \(D\) denote its
total division count. Any integer \(x\) that follows the parity pattern
\(\omega\) satisfies
\[
3^k x + c \equiv 0 \pmod{2^D}
\]
for some integer \(c\). Thus \(c\) is uniquely determined modulo \(2^D\),
and all admissible constants differ by integer multiples of \(2^D\).

To remove this ambiguity, we select the unique representative
\(c(\omega)\) satisfying
\[
0 \le c(\omega) < 2^D.
\]
This choice is canonical and depends only on the parity word \(\omega\).

\subsection*{Representative Independence}

The value \(c(\omega)\) is independent of the choice of integer
representative \(x\) within the residue class induced by \(\omega\).
Indeed, if \(x\) and \(x'\) belong to the same class modulo \(2^D\), then
\[
3^k x \equiv 3^k x' \pmod{2^D},
\]
yielding the same canonical value of \(c(\omega)\).

\subsection*{Explicit Computation (Step-by-Step)}

Given a parity word \(\omega\) of length \(k\) and total division count \(D\), 
the canonical constant \(c(\omega)\) may be computed as follows:

\begin{enumerate}
\item Choose any integer \(x\) belonging to the residue class induced by
      the parity word \(\omega\).
\item Compute \(y = 3^k x\).
\item Let \(m\) be the smallest integer greater than or equal to \(y / 2^D \).
\item Define
      \[
      c(\omega) = m \cdot 2^D - y.
      \]
\end{enumerate}

By construction, \(0 \le c(\omega) < 2^D\), and the resulting value
satisfies
\[
T_\omega(x) = \frac{3^k x + c(\omega)}{2^D} \in \mathbb{Z}.
\]
This value is independent of the chosen representative \(x\) and
provides a concrete realization of the affine map associated with \(\omega\).

\subsection*{Concrete Example}

Consider the parity word
\(\omega = (1,1,1,4)\) with \(k=4\) and total division count
\(D = 1 + 1 + 1 + 4 = 7\).

\begin{enumerate}
\item Choose a representative \(x = 15\) in the corresponding residue class.
\item Compute \(y = 3^4 \cdot 15 = 81 \cdot 15 = 1215\).
\item Find the smallest \(m\) such that \(m \cdot 2^7 \ge 1215\):
      \(2^7 = 128\), so \(m = \lceil 1215/128 \rceil = 10\).
\item Compute \(c(\omega) = 10 \cdot 128 - 1215 = 1280 - 1215 = 65\).
\end{enumerate}

Thus, the canonical affine map for this parity word is
\[
T_\omega(x) = \frac{3^4 x + 65}{2^7} = \frac{81x + 65}{128}.
\]

Checking another representative \(x = 143\) in the same residue class:
\[
T_\omega(143) = \frac{81 \cdot 143 + 65}{128} = \frac{11583 + 65}{128} = 91,
\]
confirming that the same \(c(\omega)\) works uniformly across the residue class.


\newpage
\section{Extremal Local Growth Analysis}

\begin{lemma}[Local Growth Bounds]
\label{lem:local-growth-bounds}
For each admissible local pattern occurring within a $41$-cycle,
the ratio of the total sum of entries after applying the pattern
to the total sum before applying it is bounded above as follows,
uniformly over all admissible placements of delayed doubling within
the pattern:
\[
\begin{array}{c|c}
\text{Pattern} & \text{Max Growth Factor} \\
\hline
s,d & 3 \\
d,d & 123/33 \\
s,d,s & 341/131 \\
d,d,s & 329/123
\end{array}
\]
\end{lemma}

\begin{proposition}[Extremal 41-Cycle Growth]
The maximal geometric growth rate achievable by any admissible infinite
word under the horizontal-sum dynamics ~\cite{Winkler2017} is bounded by
\[
\rho_{41}
=
\Bigl(
3^{17}
\cdot (123/33)^7
\cdot (341/131)^{10}
\cdot (329/123)^7
\Bigr)^{1/41}
< 3,
\]
where the exponents count the occurrences of each local pattern within a
$41$-cycle.
\end{proposition}

\noindent
Here the exponents record the total number of coefficients generated by each
admissible local pattern within a single $41$--cycle; the enumeration of these
patterns and the derivation of the counts $17,7,10,7$ are given explicitly in
Appendix~\ref{app:local-growth-details}, Section~\ref{sec:admissible-patterns}.

\begin{remark}[Derivation of Local Growth Constants]
The constants shown in Lemma~\ref{lem:local-growth-bounds} are obtained
by a complete extremal analysis of the horizontal-sum dynamics governing
$\widetilde A(i)$.  Within each local configuration, elements arising from
single steps evolve according to standard Pascal-type summation, yielding
maximal growth proportional to powers of $2$, while additional entries
introduced by doubling steps are bounded by summing on top of this baseline
growth.

For each admissible local pattern occurring within a $41$-cycle, all
possible placements of delayed doubling contributions were examined, and the
largest achievable ratio of post-pattern to pre-pattern row sums was
recorded.  No assumptions beyond the combinatorial structure of the
horizontal-sum construction are used.

Full illustrative computations are provided in Appendix~\ref{app:local-growth-details}.
\end{remark}


\newpage
\section{Extremal Local Growth Calculations}
\label{app:local-growth-details}

Recall that the sequence \OEIS{A020914} has first differences given by
\OEIS{A022921}.  This increment sequence consists of repeated blocks
composed of \emph{single} steps ($s$) and \emph{double} steps ($d$).

\subsection{Cycle Structure and Net Multiplicative Effect}

A frequently occurring block is the $7$--cycle
\[
\{ s, d, s, d, s, d, d \},
\]
which contains $3$ singles and $4$ doubles.  The corresponding number of
factors of $2$ is
\[
3 + 2(4) = 11,
\]
so the net multiplicative contribution of this cycle satisfies
\[
\frac{2^{11}}{3^7} < 1.
\]
Thus the $7$--cycle is strictly contractive.

This $7$--cycle is often followed by a $5$--cycle
\[
\{ s, d, s, d, d \},
\]
which contains $2$ singles and $3$ doubles, contributing
\[
2 + 2(3) = 8
\]
factors of $2$.  Since
\[
\frac{2^8}{3^5} > 1,
\]
the $5$--cycle is expansive.

Concatenating the $7$--cycle and $5$--cycle produces a $12$--cycle
\[
\{ s, d, s, d, s, d, d, s, d, s, d, d \},
\]
containing $5$ singles and $7$ doubles.  The total number of factors of $2$ is
\[
5 + 2(7) = 19,
\]
and hence
\[
\frac{2^{19}}{3^{12}} < 1.
\]
Therefore, the $12$--cycle is again contractive.

Repeated $12$--cycles diverge multiplicatively from unity until it becomes
possible to append an additional $5$--cycle, producing a $41$--cycle composed
of three consecutive $12$--cycles followed by a $5$--cycle.  The total number
of factors of $2$ in a $41$--cycle is
\[
3(19) + 8 = 65,
\]
and since
\[
\frac{2^{65}}{3^{41}} > 1,
\]
the $41$--cycle is expansive.

Importantly, a sequence composed entirely of consecutive $41$--cycles
strictly dominates the growth of \OEIS{A186009}, which also admits longer
$53$--cycles (four $12$--cycles followed by a $5$--cycle).  Consequently, any
asymptotic growth bound obtained using only $41$--cycles provides an upper
bound for the growth rate of \OEIS{A186009} itself.

\subsection*{Horizontal-Sum Dynamics and Pascal-Type Growth}

Due to doubling events, the usual $2\times$ growth limitation of a standard
Pascal triangle does not apply directly.  In particular, the sum of earlier
row entries may exceed the final entry of the row.  However, single steps do
not increase the length of the generating tuple.  As a result, growth
associated with singles follows standard Pascal-type summation rules, while
doubling events introduce additional entries whose contributions must be
handled separately.

For any single step, the generating terms are strictly increasing from left
to right across the row.  The maximal growth therefore arises from combining
Pascal-type summation and doubling events as per Lemma~\ref{lem:single-double-bound}.

\newpage
\subsection*{Admissible Local Patterns}
\label{sec:admissible-patterns}

Within a $41$--cycle, only the following four local configurations can occur.
This is a consequence of the cycle structure described above.  Each $41$--cycle
is composed of repeated $12$--cycles and a terminal $5$--cycle, where the only
step patterns are
\[
\{ s,d,s,d,s,d,d \} \quad \text{and} \quad \{ s,d,s,d,d \}.
\]
In particular, no two single steps can occur consecutively, while double steps
may occur either singly or in pairs.  Examining all contiguous subwords of these
cycles therefore yields exactly four admissible local patterns:
\[
(s,d),\quad (d,d),\quad (s,d,s),\quad (d,d,s),
\]
and no others.

Enumerating the number of occurrences of each pattern within a $41$--cycle gives
the following counts:
\[
\begin{array}{c|c}
\text{Pattern} & \text{Associated coefficient count} \\
\hline
s,d & 3(5) + 2 = 17 \\
d,d & 3(2) + 1 = 7 \\
s,d,s & 3(3) + 1 = 10 \\
d,d,s & 3(2) + 1 = 7
\end{array}
\]

\subsection{Illustrative Extremal Computations}

The generating rows evolve by a horizontal--sum rule:
\[
a_{n+1}(i+1) \;=\; a_{n+1}(i) + a_n(i+1),
\]
so that sequences of single steps exhibit ordinary Pascal--type growth.
Double steps differ only in that the terminal entry of the row is duplicated,
causing the final term to appear twice in the generating tuple while preserving
monotonicity of the preceding entries.

The following examples are not special cases but representative extremal probes
of the same underlying recurrence.  Since the horizontal--sum recurrence and
Lemma~\ref{lem:single-double-bound} provide a uniform inequality valid for all
admissible generating rows, any concrete realization that saturates these
inequalities yields an upper bound that applies to every occurrence of the same
local pattern.

We illustrate the bounding procedure with concrete examples.  Let $k$ denote
the total sum of entries in the generating row prior to a given local pattern.
By Lemma~\ref{lem:single-double-bound}, each single-to-double transition is
bounded which provides the basis for the extremal analysis:

\paragraph{Example 1.}
\[
\begin{array}{l}
\text{Single } a(10) = \underbrace{173}_{\text{sum} := k}:
\underbrace{1,7,25,55,85}_{k}
\end{array}
\]

\[
\begin{array}{l}
\text{Double } a(11) = \underbrace{476}_{\text{sum} = 3k}:
\underbrace{1,8,33,88}_{k}
\underbrace{173}_{k},
\underbrace{173}_{k}
\end{array}
\]

\[
\begin{array}{l}
\text{Single } a(12) = \underbrace{961}_{\text{sum} = 9k}:
\underbrace{1,9,42,130}_{2k}
\underbrace{303}_{3k},
\underbrace{476}_{4k}
\end{array}
\]

\[
\begin{array}{l}
\text{Double } a(13) = \underbrace{2652}_{\text{sum} = 33k}:
\underbrace{1,10,52,182}_{4k},
\underbrace{485}_{7k},
\underbrace{961}_{11k},
\underbrace{961}_{11k}
\end{array}
\]

\[
\begin{array}{l}
\text{Double } a(14) = \underbrace{8045}_{\text{sum} = 123k}:
\underbrace{1,11,63,245}_{8k},
\underbrace{730}_{15k},
\underbrace{1691}_{26k},
\underbrace{2652}_{37k},
\underbrace{2652}_{37k}
\end{array}
\]

\[
\begin{array}{l}
\text{Single } a(15) = \underbrace{17637}_{\text{sum} = 329k}:
\underbrace{1,12,75,320}_{16k},
\underbrace{1050}_{31k},
\underbrace{2741}_{57k},
\underbrace{5393}_{94k},
\underbrace{8045}_{131k}
\end{array}
\]

From these bounds, the maximal local ratios are
\[
\begin{array}{c|c}
\text{Pattern} & \text{Max Ratio} \\
\hline
s,d & 3 \\
d,d & 123/33 \\
s,d,s & 3 \\
d,d,s & 329/123
\end{array}
\]

\newpage
\paragraph{Example 2.}
\[
\begin{array}{l}
\text{Single } a(12) = \underbrace{961}_{\text{sum} := k}:
\underbrace{1,9,42,130,303,476}_{k}
\end{array}
\]

\[
\begin{array}{l}
\text{Double } a(13) = \underbrace{2652}_{\text{sum} = 3k}:
\underbrace{1,10,52,182,485}_{k},
\underbrace{961}_{k},
\underbrace{961}_{k}
\end{array}
\]

\[
\begin{array}{l}
\text{Double } a(14) = \underbrace{8045}_{\text{sum} = 13k}:
\underbrace{1,11,63,245,730}_{2k},
\underbrace{1691}_{3k},
\underbrace{2652}_{4k},
\underbrace{2652}_{4k}
\end{array}
\]

\[
\begin{array}{l}
\text{Single } a(15) = \underbrace{17637}_{\text{sum} = 37k}:
\underbrace{1,12,75,320,1050}_{4k},
\underbrace{2741}_{7k},
\underbrace{5393}_{11k},
\underbrace{8045}_{15k}
\end{array}
\]

\[
\begin{array}{l}
\text{Double } a(16) = \underbrace{51033}_{\text{sum} = 131k}:
\underbrace{1,13,88,408,1458}_{8k},
\underbrace{4199}_{15k},
\underbrace{9592}_{26k},
\underbrace{17637}_{41k},
\underbrace{17637}_{41k}
\end{array}
\]

\[
\begin{array}{l}
\text{Single } a(17) = \underbrace{108950}_{\text{sum} = 341k}:
\underbrace{1,14,102,510,1968}_{16k},
\underbrace{6167}_{31k},
\underbrace{15759}_{57k},
\underbrace{33396}_{98k},
\underbrace{51033}_{139k}\end{array}
\]

From these bounds, the maximal local ratios are
\[
\begin{array}{c|c}
\text{Pattern} & \text{Max Ratio} \\
\hline
s,d & 3 \\
d,d & 13/3 \\
s,d,s & 341/131 \\
d,d,s & 37/13
\end{array}
\]

\subsection*{Improved Extremal Alignment}

Because the total multiplicative growth over a cycle is the product of the local
growth factors, the geometric mean over one full cycle is maximized by selecting
the maximal admissible factor at each local pattern occurrence. Consequently,
once a uniform upper bound has been established for each admissible local pattern,
the extremal global growth rate over a complete cycle is obtained by multiplying
these maximal local factors according to their frequencies and taking the
corresponding geometric mean.

Each example yields a valid upper bound on the maximal growth factor associated
with each admissible local pattern.  Since these bounds arise from the same
uniform recurrence and apply globally to all admissible configurations, the
true maximal growth factor for a given pattern is bounded above by the minimum
of all such valid upper bounds.

Since the recurrence is monotone with respect to the placement of delayed doubling,
no other admissible configuration can exceed these extremal realizations.
Accordingly, combining the sharpest bounds obtained across different initiating
rows yields the tightest global envelope on admissible local growth.
\[
\begin{array}{c|c|c}
\text{Pattern} & \text{Max Ratio} & \text{Associated coefficient count} \\
\hline
s,d & 3 & 17 \\
d,d & 123/33 & 7 \\
s,d,s & 341/131 & 10 \\
d,d,s & 329/123 & 7
\end{array}
\]

Consequently, the maximal geometric growth rate of a $41$--cycle satisfies
\[
\Bigl(
3^{17}
\cdot (123/33)^7
\cdot (341/131)^{10}
\cdot (329/123)^7
\Bigr)^{1/41}
< 2.95 < 3.
\]

This bound applies \emph{a fortiori} to \OEIS{A186009}, whose admissible growth
patterns are strictly dominated, in the sense of per--symbol multiplicative
growth, by those of a pure $41$--cycle concatenation.

\paragraph{Conclusion.}
Since every admissible growth pattern of \OEIS{A186009} is composed of
$12$--cycles and $5$--cycles, and since concatenations of $41$--cycles
maximize per--symbol multiplicative growth among all such admissible
concatenations, the bound obtained above for the $41$--cycle applies
\emph{a fortiori} to \OEIS{A186009} itself.


\newpage
\bibliographystyle{unsrtnat}
\bibliography{src/latex/references}

\end{document}  
