\documentclass[12pt, oneside]{article}   	% use "amsart" instead of "article" for AMSLaTeX format
%\documentclass[12pt, oneside, draft]{article}   	% use "amsart" instead of "article" for AMSLaTeX format

%%%%%%%%%%%%%%%%%%%%%%%%%%%%%%%%%%%%%%%%%%%%%%%%%
%
% The TexLive distribution is stored in /usr/local/texlive/2020
%
% The definitions of the packages are in /usr/local/texlive/2020/texmf-dist/tex/latex
%
%%%%%%%%%%%%%%%%%%%%%%%%%%%%%%%%%%%%%%%%%%%%%%%%%

\usepackage{xparse}					% Allows for the creation of \NewDocumentCommand shortcuts

\usepackage{geometry}				% See geometry.pdf to learn the layout options. There are lots.
\geometry{letterpaper}				% ... or a4paper or a5paper or ... 
\usepackage[parfill]{parskip}		% Activate to begin paragraphs with an empty line rather than indent
\usepackage{float}					% Float package to exert more control over graphic placement
	
%\usepackage{verbatim}				% Creates a verbatim environment for code / program output
\usepackage{graphicx}				% Use pdf, png, jpg, or eps§ with pdflatex; use eps in DVI mode
									% TeX will automatically convert eps --> pdf in pdflatex
\graphicspath{{./figures/}}         

% Math packages
\usepackage{amssymb}				% Mathematical symbols
\usepackage{amsmath}				% Added for typesetting mathematical formulae
\usepackage{amsthm}					% Added for typesetting mathematical theorems
\usepackage{mathalpha}				% Added for some special characters like Q, Z and N

% Table packages in addition to built in tabular
\usepackage{booktabs}				% Adds extra commands to tabular like \toprule
\usepackage{afterpage}				% For pagination support
\usepackage{longtable}				% Support for longer tables

% Hyperlinks
\usepackage{hyperref}				% For creating internal and HTTP links

% References
\usepackage[numbers]{natbib}        % For bibliography management

% For Tikz graphics
\usepackage{tikz}					% Package for pgf and TikZ
\usetikzlibrary{shapes,arrows,chains}
\tikzstyle{startstop} = [very thick, rectangle, rounded corners, minimum width=2.5cm, minimum height=0.8cm, text centered, draw=black, text width=2.5cm]
\tikzstyle{action} = [rectangle, rounded corners, minimum width=3cm, minimum height=0.8cm, text centered, draw=black, text width=2.5cm]
\tikzstyle{decision} = [diamond, minimum width=3cm, minimum height=3cm, text centered, draw=black, text width=2cm]
\tikzstyle{arrow} = [thick,->,>=stealth]

% For theorems and definitions - each creates it's own counter which incrementa each use
\theoremstyle{definition}
\newtheorem{proposition}{Proposition}[section]
\newtheorem{define}{Definition}[section]
\newtheorem{axiom}{Axiom}[section]
\newtheorem{theorem}{Theorem}
\newtheorem{lemma}[theorem]{Lemma}
\newtheorem{corollary}[theorem]{Corollary}
\newtheorem*{remark}{Remark}

%SetFonts


\title{Global Descent and Convergent Classes in the Collatz Conjecture}
\author{Wayne Brassem}
%\date{}							% Activate to display a given date or no date


\begin{document}

% Document define commands for commonly used items in math mode
\NewDocumentCommand{\setN}{}{\mathbb{N}}				% Set of positive integers, excluding 0
\NewDocumentCommand{\setNo}{}{\mathbb{N}_0}				% Set of positive integers, including 0
\NewDocumentCommand{\setNeven}{}{\mathbb{N}_{even}}		% Set of even positive integers, excluding 0
\NewDocumentCommand{\setNodd}{}{\mathbb{N}_{odd}}		% Set of odd positive integers
\NewDocumentCommand{\setZ}{}{\mathbb{Z}}				% Set of all integers, excluding zero
\NewDocumentCommand{\setZo}{}{\mathbb{Z}_0}				% Set of all integers, including zero
\NewDocumentCommand{\setQ}{}{\mathbb{Q}}				% Set of all rational numbers

% Document define commands for commonly used items not in math mode
\NewDocumentCommand{\Rarr}{}{\textrightarrow{}}			% Right arrow

% Document commands which provide hyperlinks to OEIS sequences referenced herein
\NewDocumentCommand{\OEIS}{m}{\href{https://oeis.org/#1}{#1}}

\maketitle



\begin{abstract}

%Abstract - goals of the paper
We study the accelerated Collatz map

\[
f(x) = \begin{cases}
\displaystyle
    \phantom{3x} \frac{x}{2} & \text {if } x \text{ is even}, \\[8pt]  % Insert some spacing between rows
\displaystyle
    \frac{3x+1}{2}           & \text {if } x \text{ is odd}. \\
\end{cases}
\]

and analyze the structure of integer trajectories through parity-encoded residue classes. Each finite
parity word induces a non-overlapping residue class modulo a power of two, and convergent parity words
correspond to classes whose elements reach a strictly smaller integer after finitely many iterations.

Using explicit enumeration results for parity words and their associated division counts, we show that
convergent residue classes admit well-defined asymptotic densities. These classes may be grouped by
minimal stopping time without overlap, yielding a cumulative density that increases monotonically with
stopping time. We prove that the complement of this union has asymptotic density tending to zero,
implying that the cumulative density of convergent classes converges to one.

As a consequence, every positive integer belongs to a convergent residue class of finite stopping time
and therefore eventually reaches a smaller positive integer under iteration of the Collatz map. This
establishes global descent and rules out the existence of nontrivial cycles in the positive integers,
leaving the classical 4\Rarr2\Rarr1\Rarr4 cycle as the unique periodic orbit.

\end{abstract}
\newpage


\tableofcontents
\newpage


\section{Introduction}

\subsection{The Collatz Mapping and Known Results}

The Collatz conjecture concerns the behavior of the (accelerated) map:

\begin{define}
\label{def:accelerated-map}
The \emph{accelerated Collatz map} is the function
\( f : \setN \rightarrow \setN \) defined by
\[
f(x) = \begin{cases}
\displaystyle
    \phantom{3x} \frac{x}{2} & \text {if } x \text{ is even}, \\[8pt]
\displaystyle
    \frac{3x+1}{2}           & \text {if } x \text{ is odd}. \\
\end{cases}
\]
This map acts on the positive integers and combines each odd step of the classical Collatz iteration
\(x \mapsto 3x+1\) with the immediately following division by two
\cite{Everett1977}.
\end{define}

The Collatz conjecture asserts that, under iteration of \(f\), every positive integer eventually
reaches the cycle \(4 \to 2 \to 1 \to 4\) \cite{Collatz1937}.

Despite its elementary formulation, the conjecture has resisted proof for decades
\cite{Lagarias2010,Terras1976}. Extensive computational verification confirms convergence
for all integers up to very large bounds \cite{OliveiraESilva2010}, but such verification
does not constitute a proof and offers limited insight into the global structure
of Collatz trajectories.

Many prior approaches focus on bounding individual trajectories or analyzing
probabilistic heuristics \cite{Wirsching1998}. While these methods provide
valuable intuition, they do not address the problem at the level of the entire
positive integer space.

\subsection{Strategy of the Proof}

This paper adopts a structural approach based on decomposing the
positive integers into arithmetic residue classes determined by parity
patterns under the accelerated Collatz map.

The central idea is to study finite Collatz paths abstractly, independent
of their starting values, and to associate each such path with a
pairwise disjoint residue class of integers that follow the same parity
pattern for a fixed number of odd steps. Each class is assigned a
minimal stopping time, defined as the first iterate at which its
elements strictly decrease.

The proof proceeds through the following steps:
\begin{itemize}
\item Encode Collatz trajectories using parity words that record the
      number of divisions by two between odd steps.
\item Show that each admissible parity word induces an arithmetic
      residue class of positive integers.
\item Prove that convergent residue classes of fixed stopping time are
      pairwise disjoint.
\item Compute the asymptotic density of each class using its associated
      total division count.
\item Sum the densities of all convergent classes up to a given stopping
      time and show that the complement of this union has density
      tending to zero.
\item Conclude that every positive integer eventually reaches a smaller
      value, establishing global descent and ruling out non-trivial
      cycles.
\end{itemize}

By focusing on exhaustion of the positive integers via convergent
structures rather than on individual trajectories, this approach
provides a global mechanism for descent under the Collatz map.


\section{Definitions and Preliminaries}

\subsection{Stopping Time}

\begin{define}
The stopping time \( \sigma(x) \) of a positive integer \(x\) is the minimal integer
\(k \in \setN\) such that
\[
f^{k}(x) < x.
\]
Stopping time has been studied in the context of the $3x+1$ problem
\cite{Terras1976}.
\end{define}

\subsection{Convergent Subsequences}

\begin{define}
A \emph{convergent subsequence of length \(k\)} is a finite sequence
\[
x, f(x), f^2(x), \ldots, f^k(x)
\]
such that \( f^k(x) < x \) and \(k\) is minimal. Finite subsequences of this type were
considered in the context of the accelerated Collatz map by Everett \cite{Everett1977}.
\end{define}

\subsection{Residue Classes}

\begin{define}
Two positive integers \(x,y \in \setN\) are said to be in the same
\emph{convergent residue class of stopping time \(k\)} if their first
\(k\) Collatz iterates are identical. The idea of encoding trajectories via residue classes
or arithmetic progressions has appeared in prior works on symbolic and structural approaches
to the $3x+1$ problem \cite{Wirsching1998,Chamberland2003}.
\end{define}

\begin{define}[Convergent Residue Class]
Fix \(k \in \setN\) and \(x \in \setN\). The \emph{convergent residue class of stopping time \(k\)}
containing \(x\) is
\[
[x]_k := \{ y \in \setN \mid f^i(y) = f^i(x) \text{ for } i = 0, 1, \dots, k-1 \},
\]
where \(f\) is the accelerated Collatz map.  \([x]_k\) collects all positive integers whose
first \(k\) Collatz iterates are identical to those of \(x\).
\end{define}


\section{Division Counts and Path Encoding}

The structure of convergent residue classes is governed by the total number
of divisions by two incurred along a Collatz path. This quantity depends
only on the parity pattern of the path and not on the specific starting
integer.

\subsection{Total Division Count}

Let a Collatz path contain exactly \(k\) odd steps (``up-legs''). Between
successive odd steps, the accelerated Collatz map applies a finite number
of divisions by two. The cumulative number of such divisions determines
the arithmetic structure of the corresponding residue class.

\begin{define}[Total Division Count]
Let \(k \ge 1\). Define \(A020914(k)\) to be the total number of divisions
by two applied along an accelerated Collatz path with exactly \(k\)
odd steps.
\end{define}

This quantity coincides with the number of binary digits of \(3^k\),
as tabulated by \OEIS{A020914}, but its role here is purely arithmetic:
it records the total exponent of 2 accumulated in the denominator of the
iterated map, a perspective implicit in earlier analyses of stopping times
and parity sequences \cite{Terras1976,Wirsching1998}.

\subsection{Residue Class Modulus}

The correspondence between fixed parity patterns and arithmetic residue
classes modulo powers of two is well known in the study of Collatz dynamics
\cite{Wirsching1998,Chamberland2003}.

Fix a parity pattern with exactly \(k\) odd steps and total division count
\(A020914(k)\). All integers realizing this parity pattern form an
arithmetic progression with common difference \(2^{A020914(k)}\).

\begin{lemma}
Let \(x,y \in \setN\) follow the same Collatz parity pattern of length \(k\).
Then
\[
x \equiv y \pmod{2^{A020914(k)}}.
\]
\end{lemma}

\begin{proof}
Each odd step contributes a factor of 3 in the numerator, while each even
step contributes a factor of 2 in the denominator. After \(k\) odd steps,
the accumulated denominator is exactly \(2^{A020914(k)}\), independent of
the starting integer, which fixes the residue class modulo this power of two.
\end{proof}
\subsection{Density of Convergent Classes}

\begin{proposition}
Each convergent residue class with \(k\) odd steps has asymptotic density
\[
2^{-A020914(k)}.
\]
\end{proposition}

\begin{proof}
Each class is an arithmetic progression with step size \(2^{A020914(k)}\)
and exactly one representative per period. The result follows.
\end{proof}

\subsection{Example}

For convergent subsequences with \(k=4\) odd steps, \(A020914(4)=7\).
Thus all associated residue classes consist of integers of the form
\[
x = 2^7 n + r, \quad n \in \setN,
\]
where the residue \(r\) depends on the specific parity pattern.
The spacing between consecutive elements is therefore \(2^7 = 128\).

The preceding example illustrates that convergent behavior depends only
on parity structure and accumulated division count, motivating an abstract
description of Collatz trajectories independent of their starting values.


\section{Grammar of Collatz Parity Words}

\subsection{Parity Words}

\begin{remark}[Parity Grammar as a Dynamical Encoding]
The parity-word grammar introduced in this section provides a symbolic
encoding of accelerated Collatz trajectories. This approach, introduced
in the context of $3x+1$ dynamics by Chamberland \cite{Chamberland2003}, is
not injective: many integers correspond to the same parity word, forming
arithmetic residue classes. Our use of parity words differs in that we
exploit them to rigorously quantify densities of convergent residue
classes and establish global descent. 

Each parity word induces an affine transformation whose linear
coefficient determines whether all sufficiently large integers in the
associated residue class decrease after a fixed number of odd steps.
Thus, while individual trajectories are collapsed under the encoding,
their asymptotic behavior with respect to growth and descent is retained.
This suffices for density, exhaustion, and cycle-exclusion arguments.
\end{remark}

\begin{define}[Parity Word]
A \emph{parity word} of length \(k \ge 1\) is a finite sequence
\[
\omega = (d_1, d_2, \dots, d_k),
\]
where each \(d_i \in \setN\) denotes the number of divisions by two
applied after the \(i\)-th odd step of an accelerated Collatz trajectory.
\end{define}

Each parity word determines a linear-affine transformation
\[
x \mapsto \frac{3^k x + c(\omega)}{2^{d_1+\cdots+d_k}},
\]
where \(c(\omega)\) is a canonical constant uniquely determined by the
parity word (see Appendix~\ref{app:canonical-c}).

\begin{remark}[Parity Words Encode Structure, Not Values]
Fixing a parity word $\omega$ determines the sequence of arithmetic
operations applied along an accelerated Collatz trajectory. Consequently,
both the linear coefficient $3^k/2^{d_1+\cdots+d_k}$ and the affine term
$c(\omega)$ depend only on $\omega$ and not on the starting integer.

Thus, convergence or divergence is a property of the parity word itself,
and applies uniformly to all sufficiently large elements of the induced
residue class.
\end{remark}

\begin{define}[Admissible Parity Word]
A parity word \(\omega\) is \emph{admissible} if there exists a positive
integer \(x\) realizing \(\omega\) such that \(T_\omega(x) \in \setN\).
\end{define}

\begin{lemma}[Minimal Division Count]
For any admissible parity word of length \(k\),
\[
d_1 + \cdots + d_k \ge A020914(k),
\]
where \(A020914(k)\) is the minimal power of two required to clear
denominators in an accelerated Collatz path with \(k\) odd steps.
\end{lemma}

\begin{define}[Convergent Parity Word]
An admissible parity word \(\omega = (d_1,\dots,d_k)\) is
\emph{convergent} if
\[
\frac{3^k}{2^{d_1+\cdots+d_k}} < 1.
\]
\end{define}

\begin{lemma}[Associated Collatz Map]
\label{lem:associated-map}
Each parity word \(\omega = (d_1,\dots,d_k)\) induces an affine map
\[
T_\omega(x) = \frac{3^k x + c(\omega)}{2^{d_1+\cdots+d_k}},
\]
where \(c(\omega)\) is the unique integer satisfying
\[
0 \le c(\omega) < 2^{d_1+\cdots+d_k}
\]
for which \(T_\omega(x)\in\setN\) for all integers \(x\) realizing \(\omega\).
\end{lemma}

\begin{proof}
Each odd step contributes a factor of \(3\) to the numerator, while each
division by two contributes a factor of \(2\) to the denominator. The
constant term accumulates from the repeated application of the
\(3x+1\) operation and depends only on the parity pattern.
\end{proof}

The constant \(c(\omega)\) is uniquely determined modulo \(2^D\);
a canonical representative in the range \(0 \le c(\omega) < 2^D\)
and an explicit computation are given in
Appendix~\ref{app:canonical-c}.

\begin{remark}[Affine Map Representation of Collatz Trajectories]
For any integer realizing a parity word $\omega$, the value $T_\omega(x)$
coincides exactly with the result of applying the accelerated Collatz map
for $k$ successive odd steps. Thus the affine map $T_\omega$ represents a
compression of the corresponding finite Collatz trajectory into a single
step.
\end{remark}

\begin{lemma}
\label{lem:residue_class}
Each convergent parity word \(\omega\) induces a residue class
\[
x \equiv r_\omega \pmod{2^{A020914(k)}},
\]
whose elements all follow the same Collatz parity pattern for \(k\) odd steps and reach
an integer strictly smaller than the starting value after at most \(k\) odd steps.
\end{lemma}

\begin{proposition}
The number of convergent parity words of length \(k\) is given by
\OEIS{A186009}.
\end{proposition}


\section{Cumulative Density of Convergent Classes}

\subsection{Asymptotic Control of Density Contributions}

Define
\[
\widetilde A(i) := A186009(i+1),
\]
so that $\widetilde A(i)$ counts convergent residue classes of stopping time $i$.

Let
\[
B(i) := A020914(i),
\]
so that the density contribution of convergent residue classes of stopping time $i$ is
\[
\widetilde A(i)\, 2^{-B(i)}.
\]

From the structure of $A020914$, we know:
\begin{itemize}
    \item $B(i)$ is strictly increasing, with increments either 1 or 2 and no zeros.
    \item Asymptotically, $B(i) \sim (\log_2 3)\, i$.
    \item By construction, $2^{B(i)} > 3^i$ for all $i \ge 0$.
\end{itemize}

Because $2^{B(i)}$ is a power of two and $B(i)$ is defined as the minimal
integer such that $2^{B(i)} > 3^i$, it follows that $2^{B(i)}$ is the smallest
power of two exceeding $3^i$. Since consecutive powers of two differ by a
factor of $2$, we obtain the sharp bound
\[
3^i < 2^{B(i)} < 2 \cdot 3^i.
\]

\begin{remark}[Historical Motivation and True Growth Delay]
A subtle but important distinction must be made between index offset and
effective growth delay.

The sequence \(A020914(n)\) is indexed starting at \(n=0\), while
\(A186009(n)\) is indexed starting at \(n=1\). The first doubling increment in
\(A020914\) occurs at index \(1\), whereas its effect on the numerator
\(A186009\) appears at term \(n=4\). This corresponds to an index displacement
of \(4-1=3\).

However, the effective delay in multiplicative growth is only two steps. The
remaining unit of offset arises solely from differing indexing conventions
and does not correspond to a missed multiplicative contribution. Consequently,
the numerator realizes each doubling increment after a fixed finite delay of
two growth steps relative to the denominator and is the only delay relevant
for asymptotic growth comparisons.

Here and throughout, the index $i$ refers to the stopping time (equivalently,
the parity-word length); the differing OEIS indexing conventions of
$A020914$ and $A186009$ are the motivation for defining $\widetilde A(i)$.

\end{remark}

\noindent
\textbf{Conclusion 1 (constant factor control).}
This bound depends only on the definition of $A020914$ and does not rely on
any delayed-doubling or numerator-growth arguments. Consequently, the reciprocal satisfies
\[
2^{-B(i)} \in \left( \frac{1}{2 \cdot 3^i}, \frac{1}{3^i} \right),
\]
giving a precise constant factor for the density contribution at level $i$.

We now record this observation formally for later reference.
\begin{lemma}[Asymptotic Control of Density Contributions]
\label{lem:asymptotic-control}
For all $i \ge 0$,
\[
3^i < 2^{A020914(i)} < 2\cdot 3^i.
\]
Consequently, the density contribution of convergent residue classes of
stopping time $i$ satisfies
\[
\frac{\widetilde A(i)}{3^i}
\;<\;
\widetilde A(i)\,2^{-A020914(i)}
\;<\;
\frac{\widetilde A(i)}{2 \cdot 3^i}.
\]
\end{lemma}

\begin{proof}
By definition, $A020914(i)$ is the minimal integer $m$ such that $2^m > 3^i$.
Consecutive powers of two differ by a factor of $2$, so
\[
3^i < 2^{A020914(i)} < 2 \cdot 3^i.
\]
Taking reciprocals and multiplying by $\widetilde A(i)$ gives the stated bounds.
\end{proof}

\subsection{Delayed Dominance of the Denominator}

\begin{lemma}[Delayed Dominance of the Denominator]
\label{lem:delay-dominance}
Let $a_i := \widetilde A(i)$ and $b_i := A020914(i)$.  
Both sequences are generated from the same increment sequence
$\epsilon_i \in \{1,2\}$ encoded by \OEIS{A022921}, where
\[
b_i = \sum_{k=1}^i \epsilon_k.
\]

Each increment $\epsilon_k$ contributes a multiplicative factor
$2^{\epsilon_k}$ to the denominator immediately, while its effect
on the numerator $a_i$ is realized only after a fixed delay of two
steps when it is incorporated into the horizontal-sum construction.

As a result, the denominator accumulates an additional factor of $2^i$
that has no delayed counterpart in the numerator. Consequently, 
\[
\frac{a_i}{2^{b_i}} = \frac{a_i}{2^{i + D_i}} \to 0
\quad \text{as } i \to \infty.
\]
\end{lemma}

\subsection{Delayed Doubling and Upper Bound on Numerator Growth}

\begin{lemma}[Subexponential Prefix Growth]
Let $D_i$ denote the number of double events up to stopping time $i$. Then
\[
\frac{\widetilde A(i)}{2^{B(i)}} \longrightarrow 0
\quad \text{as } i \to \infty.
\]
Equivalently,
\[
\widetilde A(i) = 2^{D_i} \cdot o(2^i),
\]
because each multiplicative contribution in the numerator is realized only after
a fixed delay, while the denominator accumulates contributions immediately.
\end{lemma}

In particular, although $\widetilde A(i) / 2^{D_i}$ is unbounded, its growth rate
is strictly dominated by the factor $2^i$ contributed by the denominator.

Although numerically the residual factor $\widetilde A(i) / 2^{D_i}$ appears to grow at
a rate consistent with an exponential base strictly less than $2$, \emph{no exponential
upper bound is assumed or required}.  The conclusion follows solely from the uniform
$2^i$ phase advantage of the denominator over the numerator.

\end{lemma}

\subsubsection*{Conclusion}

The growth of both the numerator $\widetilde A(i)$ and the denominator
\(2^{B(i)}\) is governed by the same underlying increment sequence
\OEIS{A022921}, which encodes whether each step contributes one or two powers
of two. Crucially, this shared control is not symmetric: each increment in
\OEIS{A022921} is realized immediately in the denominator through the
accumulation of \(A020914(i)\), while its effect on the numerator is delayed
by a fixed offset of $2$ terms arising from the horizontal-sum (parity-word)
construction of $\widetilde A(i)$.

As a result, the denominator consistently maintains a \emph{phase advantage}
over the numerator: each multiplicative growth event is realized earlier in
\(2^{B(i)}\) than in \(\widetilde A(i)\).  This phase advantage is uniform
and irreversible: once a multiplicative contribution is realized in the
denominator, the numerator can never recover it at later stopping times.

Combined with the structural restriction on the frequency of doubling events
imposed by admissible cycle patterns, this ensures that the numerator cannot
sustain growth comparable to the denominator at large stopping times.

Consequently, the density contribution
\[
d_i := \widetilde A(i)\,2^{-A020914(i)}
\]
decreases with stopping time \(i\), and residue classes of large stopping time
become asymptotically negligible. Since $\sum_i d_i$ converges absolutely and
exhausts all admissible residue classes, the cumulative density tends to $1$.


\section{Global Descent}

\begin{theorem}[Global Descent]
Every positive integer eventually reaches a smaller positive integer
under iteration of the Collatz map.
\end{theorem}

\begin{proof}
By Corollary~\ref{cor:cum_density}, the union of all convergent residue
classes has asymptotic density 1. Hence every positive integer belongs
to some convergent residue class of finite stopping time.

For each such class, there exists a finite integer \(k\) such that
\(f^k(x) < x\). Therefore, every positive integer eventually reaches
a smaller positive integer under iteration of the Collatz map.
\end{proof}


\section{Non-Existence of Non-Trivial Cycles}

\begin{theorem}
The only cycle in the positive integers under the Collatz map is
\(4 \rightarrow 2 \rightarrow 1 \rightarrow 4\).
\end{theorem}


\section{Conclusion}

We have shown that the positive integers may be partitioned into
non-overlapping residue classes determined by finite Collatz parity
patterns, each assigned a minimal stopping time. By explicitly
enumerating convergent parity words and summing the asymptotic densities
of their associated residue classes, we proved that the cumulative
density of convergent classes converges to one.

This establishes a global descent mechanism for the accelerated Collatz
map: every positive integer eventually reaches a strictly smaller value,
ruling out the existence of non-trivial cycles. The argument is global
and structural in nature, relying neither on probabilistic heuristics
nor on bounds for individual trajectories.

The approach reframes the Collatz conjecture as an exhaustion problem
over arithmetic classes rather than a question of trajectory control,
suggesting a framework that may be applicable to related discrete
dynamical systems.


\newpage
\appendix
\section{Canonical Determination of the Constant \(c(\omega)\)}
\label{app:canonical-c}

In Lemma~\ref{lem:associated-map}, each parity word
\(\omega = (d_1,\dots,d_k)\) induces an affine Collatz map of the form
\[
T_\omega(x) = \frac{3^k x + c(\omega)}{2^{D}}, 
\qquad D = d_1 + \cdots + d_k.
\]
The constant \(c(\omega)\) is uniquely determined by the parity word
\(\omega\) up to congruence modulo \(2^D\). In this appendix, we describe
a canonical choice of representative and an explicit method for
computing it.

\subsection*{Existence and Uniqueness}

Fix a parity word \(\omega\) of length \(k\) and let \(D\) denote its
total division count. Any integer \(x\) that follows the parity pattern
\(\omega\) satisfies
\[
3^k x + c \equiv 0 \pmod{2^D}
\]
for some integer \(c\). Thus \(c\) is uniquely determined modulo \(2^D\),
and all admissible constants differ by integer multiples of \(2^D\).

To remove this ambiguity, we select the unique representative
\(c(\omega)\) satisfying
\[
0 \le c(\omega) < 2^D.
\]
This choice is canonical and depends only on the parity word \(\omega\).

\subsection*{Representative Independence}

The value \(c(\omega)\) is independent of the choice of integer
representative \(x\) within the residue class induced by \(\omega\).
Indeed, if \(x\) and \(x'\) belong to the same class modulo \(2^D\), then
\[
3^k x \equiv 3^k x' \pmod{2^D},
\]
yielding the same canonical value of \(c(\omega)\).

\subsection*{Explicit Computation (Step-by-Step)}

Given a parity word \(\omega\) of length \(k\) and total division count \(D\), 
the canonical constant \(c(\omega)\) may be computed as follows:

\begin{enumerate}
\item Choose any integer \(x\) belonging to the residue class induced by
      the parity word \(\omega\).
\item Compute \(y = 3^k x\).
\item Let \(m\) be the smallest integer such that
      \(m \cdot 2^D \ge y\).
\item Define
      \[
      c(\omega) = m \cdot 2^D - y.
      \]
\end{enumerate}

By construction, \(0 \le c(\omega) < 2^D\), and the resulting value
satisfies
\[
T_\omega(x) = \frac{3^k x + c(\omega)}{2^D} \in \mathbb{Z}.
\]
This value is independent of the chosen representative \(x\) and
provides a concrete realization of the affine map associated with \(\omega\).

\subsection*{Concrete Example}

Consider the parity word
\(\omega = (1,1,1,4)\) with \(k=4\) and total division count
\(D = 1 + 1 + 1 + 4 = 7\).

\begin{enumerate}
\item Choose a representative \(x = 15\) in the corresponding residue class.
\item Compute \(y = 3^4 \cdot 15 = 81 \cdot 15 = 1215\).
\item Find the smallest \(m\) such that \(m \cdot 2^7 \ge 1215\):
      \(2^7 = 128\), so \(m = \lceil 1215/128 \rceil = 10\).
\item Compute \(c(\omega) = 10 \cdot 128 - 1215 = 1280 - 1215 = 65\).
\end{enumerate}

Thus, the canonical affine map for this parity word is
\[
T_\omega(x) = \frac{3^4 x + 65}{2^7} = \frac{81x + 65}{128}.
\]

Checking another representative \(x = 143\) in the same residue class:
\[
T_\omega(143) = \frac{81 \cdot 143 + 65}{128} = \frac{11588 + 65}{128} = 91,
\]
confirming that the same \(c(\omega)\) works uniformly across the residue class.


\newpage
\bibliographystyle{unsrtnat}
\bibliography{src/latex/references}

\end{document}  
