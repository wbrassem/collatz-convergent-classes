\documentclass[12pt, oneside]{article}   	% use "amsart" instead of "article" for AMSLaTeX format
%\documentclass[12pt, oneside, draft]{article}   	% use "amsart" instead of "article" for AMSLaTeX format

%%%%%%%%%%%%%%%%%%%%%%%%%%%%%%%%%%%%%%%%%%%%%%%%%
%
% The TexLive distribution is stored in /usr/local/texlive/2020
%
% The definitions of the packages are in /usr/local/texlive/2020/texmf-dist/tex/latex
%
%%%%%%%%%%%%%%%%%%%%%%%%%%%%%%%%%%%%%%%%%%%%%%%%%

\usepackage{xparse}					% Allows for the creation of \NewDocumentCommand shortcuts

\usepackage{geometry}				% See geometry.pdf to learn the layout options. There are lots.
\geometry{letterpaper}				% ... or a4paper or a5paper or ... 
\usepackage[parfill]{parskip}		% Activate to begin paragraphs with an empty line rather than indent
\usepackage{float}					% Float package to exert more control over graphic placement
	
%\usepackage{verbatim}				% Creates a verbatim environment for code / program output
\usepackage{graphicx}				% Use pdf, png, jpg, or eps§ with pdflatex; use eps in DVI mode
									% TeX will automatically convert eps --> pdf in pdflatex
\graphicspath{{./figures/}}         

% Math packages
\usepackage{amssymb}				% Mathematical symbols
\usepackage{amsmath}				% Added for typesetting mathematical formulae
\usepackage{amsthm}					% Added for typesetting mathematical theorems
\usepackage{mathalpha}				% Added for some special characters like Q, Z and N

% Table packages in addition to built in tabular
\usepackage{booktabs}				% Adds extra commands to tabular like \toprule
\usepackage{afterpage}				% For pagination support
\usepackage{longtable}				% Support for longer tables

% Hyperlinks
\usepackage{hyperref}				% For creating internal and HTTP links

% References
\usepackage[numbers]{natbib}        % For bibliography management

% For Tikz graphics
\usepackage{tikz}					% Package for pgf and TikZ
\usetikzlibrary{shapes,arrows,chains}
\tikzstyle{startstop} = [very thick, rectangle, rounded corners, minimum width=2.5cm, minimum height=0.8cm, text centered, draw=black, text width=2.5cm]
\tikzstyle{action} = [rectangle, rounded corners, minimum width=3cm, minimum height=0.8cm, text centered, draw=black, text width=2.5cm]
\tikzstyle{decision} = [diamond, minimum width=3cm, minimum height=3cm, text centered, draw=black, text width=2cm]
\tikzstyle{arrow} = [thick,->,>=stealth]

% For theorems and definitions - each creates it's own counter which incrementa each use
\theoremstyle{definition}
\newtheorem{proposition}{Proposition}[section]
\newtheorem{define}{Definition}[section]
\newtheorem{axiom}{Axiom}[section]
\newtheorem{theorem}{Theorem}
\newtheorem{lemma}[theorem]{Lemma}
\newtheorem{corollary}[theorem]{Corollary}
\newtheorem*{remark}{Remark}

%SetFonts


\title{Global Descent and Convergent Classes in the Collatz Conjecture}
\author{Wayne Brassem}
%\date{}							% Activate to display a given date or no date


\begin{document}

% Document define commands for commonly used items in math mode
\NewDocumentCommand{\setN}{}{\mathbb{N}}				% Set of positive integers, excluding 0
\NewDocumentCommand{\setNo}{}{\mathbb{N}_0}				% Set of positive integers, including 0
\NewDocumentCommand{\setNeven}{}{\mathbb{N}_{even}}		% Set of even positive integers, excluding 0
\NewDocumentCommand{\setNodd}{}{\mathbb{N}_{odd}}		% Set of odd positive integers
\NewDocumentCommand{\setZ}{}{\mathbb{Z}}				% Set of all integers, excluding zero
\NewDocumentCommand{\setZo}{}{\mathbb{Z}_0}				% Set of all integers, including zero
\NewDocumentCommand{\setQ}{}{\mathbb{Q}}				% Set of all rational numbers

% Document define commands for commonly used items not in math mode
\NewDocumentCommand{\Rarr}{}{\textrightarrow{}}			% Right arrow

% Document commands which provide hyperlinks to OEIS sequences referenced herein
\NewDocumentCommand{\OEIS}{m}{\href{https://oeis.org/#1}{#1}}

\maketitle



\begin{abstract}

We study the accelerated Collatz map
\[
f(x) = \begin{cases}
\displaystyle
    \phantom{3x} \frac{x}{2} & \text {if } x \text{ is even}, \\[8pt]
\displaystyle
    \frac{3x+1}{2}           & \text {if } x \text{ is odd},
\end{cases}
\]
and analyze integer trajectories through parity-encoded residue classes.
For fixed length, each admissible parity word determines a residue class
modulo a power of two, and these classes are pairwise disjoint. Convergent
parity words correspond to classes whose elements reach a strictly smaller
integer after finitely many iterations.

Using explicit enumeration of admissible parity words and their associated
division counts, we show that each convergent residue class admits a
well-defined asymptotic density. Grouping these classes by minimal stopping
time yields, at each level, a finite disjoint family whose cumulative density
increases monotonically. We prove that the density contribution of newly
appearing convergent classes tends to zero as the stopping time grows.
Consequently, the total density of the union of convergent residue classes
approaches one.

Because the parity-induced residue classes form a partition of the positive
integers, every integer must belong to some convergent residue class of finite
stopping time and therefore eventually reaches a smaller positive integer
under iteration of the Collatz map. This establishes global descent and rules
out the existence of nontrivial cycles in the positive integers, leaving the
classical \(4 \rightarrow 2 \rightarrow 1 \rightarrow 4\) cycle as the unique
periodic orbit.

\end{abstract}


\tableofcontents
\newpage


\section{Introduction}

\subsection{The Collatz Mapping and Known Results}

The Collatz conjecture concerns the iteration of a simple piecewise-defined map
introduced by Lothar Collatz in 1937~\cite{Collatz1937}, defined as follows.

\begin{define}[Standard Collatz Map]
\label{def:standard-map}
The \emph{standard Collatz map} $g : \setN \to \setN$ is defined by
\[
g(x) =
\begin{cases}
\displaystyle
    \phantom{3x} \frac{x}{2} & \text {if } x \text{ is even}, \\[8pt]
\displaystyle
    3x+1,                    & \text{if $x$ is odd}.
\end{cases}
\]
\end{define}

In this paper, we primarily study the accelerated map defined below on the
positive integers.  All results are stated in terms of the accelerated map
\( f \), which is dynamically equivalent to the standard Collatz map \( g \).

\begin{define}[Accelerated Collatz Map]
\label{def:accelerated-map}
The \emph{accelerated Collatz map} is the function
\( f : \setN \rightarrow \setN \) defined by
\[
f(x) = \begin{cases}
\displaystyle
    \phantom{3x} \frac{x}{2} & \text {if } x \text{ is even}, \\[8pt]
\displaystyle
    \frac{3x+1}{2}           & \text {if } x \text{ is odd}. \\
\end{cases}
\]
This map acts on the positive integers and combines each odd step of the classical Collatz
iteration \(x \mapsto 3x+1\) with the immediately following division by two
\cite{Everett1977}.
\end{define}

The Collatz conjecture asserts that, under iteration of $f$, every positive
integer eventually reaches the trivial cycle $2 \to 1 \to 2$, which corresponds
to the classical $4 \to 2 \to 1 \to 4$ cycle under the standard map $g$
\cite{Collatz1937}.

Despite its elementary formulation, the conjecture has resisted proof for decades
\cite{Lagarias2010,Terras1976}. Extensive computational verification confirms convergence
for all integers up to very large bounds \cite{OliveiraESilva2010}, but such verification
does not constitute a proof and offers limited insight into the global structure
of Collatz trajectories.

Many prior approaches focus on bounding individual trajectories or probabilistic heuristics 
\cite{Wirsching1998}. While these methods provide valuable intuition, they do not directly
address the problem at the level of the entire positive integer space.


\subsection{Strategy of the Proof}

This paper adopts a structural approach based on decomposing the
positive integers into arithmetic residue classes determined by parity
patterns under the accelerated Collatz map.

The central idea is to study finite Collatz paths abstractly and to associate each such
path with a pairwise disjoint residue class of integers that follow the same parity
pattern for a fixed number of odd steps. Each class is assigned a
minimal stopping time, defined as the first iterate at which its
elements strictly decrease.

The proof proceeds through the following steps:
\begin{itemize}
\item Encode Collatz trajectories using parity words that record the
      number of divisions by two between odd steps.
\item Show that each admissible parity word induces an arithmetic
      residue class of positive integers.
\item Prove that convergent residue classes of fixed stopping time are
      pairwise disjoint.
\item Compute the asymptotic density of each class using its associated
      total division count.
\item Show that the density contribution of newly appearing convergent
      classes tends to zero as the stopping time increases, and that
      any residue classes not appearing among the convergent classes
      have vanishing asymptotic density.
\item Conclude that every positive integer eventually reaches a smaller
      value, establishing global descent and ruling out non-trivial
      cycles.
\end{itemize}

By focusing on cumulative densities of the positive integers via convergent
structures rather than on individual trajectories, this approach
provides a global and deterministic mechanism for descent under
the Collatz map.


\newpage
\section{Definitions and Preliminaries}

\subsection{Stopping Time}

\begin{define}
The \emph{stopping time} \( \sigma(x) \) of a positive integer \(x\) is the minimal integer
\(k \in \setN\) (if such a \(k\) exists) such that
\[
f^k(x) < x.
\]
Stopping time has been studied in the context of the $3x+1$ problem
\cite{Terras1976}.
\end{define}

\subsection{Convergent Segments}

\begin{define}
A \emph{convergent segment of length \(k\)} is a finite sequence
\[
x, f(x), f^2(x), \ldots, f^k(x)
\]
such that \( f^k(x) < x \) and \(k\) is minimal. Finite segments of this type were
considered in the context of the accelerated Collatz map by Everett \cite{Everett1977}.
\end{define}


\subsection{Residue Classes}

\begin{define}
Two positive integers \(x,y \in \setN\) are said to be in the same \emph{residue class of
stopping time \(k\)} if their first \(k\) Collatz iterates are identical. The idea of encoding
trajectories via residue classes or arithmetic progressions has appeared in prior works on
symbolic and structural approaches to the $3x+1$ problem \cite{Wirsching1998,Chamberland2003}.
\end{define}

\begin{define}[Residue Class of Depth \(k\)]
Fix \(k \in \setN\) and \(x \in \setN\). The \emph{residue class of depth \(k\)}
containing \(x\) is
\[
[x]_k := \{ y \in \setN \mid f^i(y) = f^i(x) \text{ for } i = 0, 1, \dots, k-1 \}.
\]
\end{define}

\begin{define}[Convergent Residue Class]
A residue class \([x]_k\) is called \emph{convergent of stopping time \(k\)} if
\[
f^k(x) < x.
\]
\end{define}


\section{Division Counts and Path Encoding}

The structure of convergent residue classes is governed by the total number
of divisions by two incurred along a Collatz path, which depends only on the
parity pattern of the path and not on the specific starting integer.

In this section we will refer to the sequence of division counts after each odd step
as a \emph{parity word}. A formal definition of parity words, together with
the admissibility conditions that govern them, appears in Section~\ref{sec:grammar}.

\subsection{Total Division Count}

Let a Collatz path contain exactly $k$ odd steps (``up-legs''). Between
successive odd steps, the accelerated Collatz map applies a finite number
of divisions by two. The cumulative number of such divisions determines
the arithmetic structure of the corresponding residue class.

\begin{define}[Total Division Count]
Let $k \ge 1$. Define $A020914(k)$ to be the minimal total number of divisions 
by two incurred along an accelerated Collatz path containing exactly $k$
odd steps.
\end{define}

Although parity words with the same number of odd steps may differ
in the distribution of divisions by two between steps, the total number
of divisions is invariant. For any Collatz path with exactly $k$ odd steps,
the sum of the division counts equals $A020914(k)$. This quantity depends
only on the number of odd steps and not on the specific parity word.

This quantity coincides with the number of binary digits of $3^k$,
as tabulated by \OEIS{A020914}, but its role here is purely arithmetic:
it records the total exponent of 2 accumulated in the denominator of the
iterated map, a perspective implicit in earlier analyses of stopping times
and parity sequences \cite{Terras1976,Wirsching1998}.


\subsection{Density of Residue Classes}

\begin{lemma}[Disjointness of Residue Classes]
For fixed $k$, the residue classes induced by distinct parity words are pairwise disjoint.
\end{lemma}

\begin{proof}
Each parity word with $k$ odd steps determines a unique residue
class of the form
\[
n \equiv r \pmod{2^{A020914(k)}},
\]
with exactly one representative per period of length $2^{A020914(k)}$, which
defines the residue class modulo $2^{A020914(k)}$. If two distinct parity words
produced the same residue $r \pmod{2^{A020914(k)}}$, then they would induce
identical arithmetic progressions and hence identical Collatz trajectories,
contradicting the uniqueness of the parity encoding. Therefore, distinct
words correspond to distinct residues modulo $2^{A020914(k)}$, and the
associated residue classes are disjoint.
\end{proof}

\begin{proposition}
Each residue class with $k$ odd steps has asymptotic density
\[
2^{-A020914(k)}.
\]
\end{proposition}

\begin{proof}
Each class is an arithmetic progression with step size $2^{A020914(k)}$
and exactly one representative per period. The result follows.
\end{proof}

\begin{remark}
At fixed $k$, there are finitely many distinct convergent residue classes.
Let $\widetilde A(k)$ denote the number of such classes. Since these classes
are pairwise disjoint, their total density contribution at stopping time $k$ is
\[
\widetilde A(k)\,2^{-A020914(k)}.
\]
\end{remark}


\subsection{Example}

For convergent segments with $k=4$ odd steps, $A020914(4)=7$.
Thus all associated residue classes consist of integers of the form
\[
x = 2^7 n + r, \quad n \in \setN,
\]
where the residue $r$ depends on the specific parity word. Each residue $r$
corresponds to exactly one parity word of length 4, so the spacing between
consecutive integers in a given class is $2^7 = 128$.

This example illustrates that convergent behavior depends only on parity
structure and accumulated division count, motivating an abstract
description of Collatz trajectories independent of their starting values.



\section{Grammar of Collatz Parity Words}
\label{sec:grammar}

\subsection{Parity Words as Dynamical Encodings}

\begin{remark}[Parity Grammar as a Dynamical Encoding]
The parity-word grammar introduced in this section provides a symbolic
encoding of accelerated Collatz trajectories, following Chamberland~\cite{Chamberland2003}.
This encoding is not injective: many integers correspond to the same parity word,
forming arithmetic residue classes.

Our use of parity words differs in that we exploit this encoding to
rigorously quantify densities of convergent residue classes and establish
global descent. While individual trajectories are collapsed under the encoding,
their expansive and contractive behavior is retained, sufficient for
density, exhaustion, and cycle-exclusion arguments.
\end{remark}


\subsection{Formal Definition of Parity Words}

\begin{define}[Parity Word]
A \emph{parity word} of length \(k \ge 1\) is a finite sequence
\[
\omega = (d_1, d_2, \dots, d_k),
\]
where each \(d_i \in \setN_{\ge 1}\) denotes the number of divisions by two
applied after the $i$-th odd step of an accelerated Collatz trajectory.
\end{define}

Each parity word determines a linear-affine transformation
\[
x \longmapsto \frac{3^k x + c(\omega)}{2^{d_1+\cdots+d_k}},
\]
where \(c(\omega)\) is a canonical integer uniquely determined by the
parity word (see Appendix~\ref{app:canonical-c}).  
This expression shows that each parity word encodes a rational linear map on integers.

\begin{remark}[Parity Words Encode Structure, Not Values]
Fixing a parity word $\omega$ determines the sequence of arithmetic
operations applied along an accelerated Collatz trajectory. Consequently,
both the linear coefficient \(3^k/2^{d_1+\cdots+d_k}\) and the affine term
\(c(\omega)\) depend only on $\omega$ and not on the starting integer.

Thus, convergence or divergence is a property of the parity word itself
and applies uniformly to all elements of the induced residue class.
\end{remark}


\subsection{Admissibility of Parity Words}

\begin{define}[Admissible Parity Word]
A parity word \(\omega\) is \emph{admissible} if there exists a positive
integer $x$ realizing $\omega$ such that the induced affine map
\[
T_\omega(x) = \frac{3^k x + c(\omega)}{2^{d_1+\cdots+d_k}}
\]
is a positive integer.  

Admissibility depends implicitly on the minimal total division count \(A020914(k)\),
ensuring that the affine map maps integers to integers.
\end{define}

\begin{lemma}[Minimal Division Count]
For any admissible parity word of length $k$,
\[
d_1 + \cdots + d_k \ge A020914(k),
\]
where $A020914(k)$ denotes the minimal total number of divisions by two
required to clear denominators in an accelerated Collatz trajectory with $k$ odd steps. Equality holds if and only if the parity word achieves admissibility with minimal total division count.
\end{lemma}

\begin{define}[Convergent Parity Word]
An admissible parity word \(\omega = (d_1,\dots,d_k)\) is
\emph{convergent} if
\[
\frac{3^k}{2^{d_1+\cdots+d_k}} < 1.
\]
\end{define}


\subsection{Residue Classes Induced by Parity Words}

Fix a parity word \(\omega\) of length \(k\) that is realizable by some positive integer \(x\). 
Let $A020914(k)$ denote the minimal total number of divisions by two along any path realizing $\omega$.

\begin{lemma}[Residue Class Modulus]
Let $x,y \in \setN$ both realize the same parity word \(\omega\). Then
\[
x \equiv y \pmod{2^{A020914(k)}}.
\]
\end{lemma}

\begin{proof}
Each odd step contributes a factor of 3 in the numerator, while each even step contributes a factor of 2 in the denominator. 
After $k$ odd steps, the accumulated denominator is exactly $2^{A020914(k)}$, independent of the starting integer. 
Hence, all integers realizing $\omega$ lie in a single residue class modulo $2^{A020914(k)}$.
\end{proof}

\begin{lemma}[Associated Collatz Map]
\label{lem:associated-map}
Each parity word \(\omega = (d_1,\dots,d_k)\) induces an affine map
\[
T_\omega(x) = \frac{3^k x + c(\omega)}{2^{d_1+\cdots+d_k}},
\]
where \(c(\omega)\) is the unique integer satisfying
\[
0 \le c(\omega) < 2^{d_1+\cdots+d_k}
\]
for which $T_\omega(x) \in \setN$ for all integers realizing $\omega$.
\end{lemma}

\begin{remark}[Affine Compression of Trajectories]
For any integer realizing a parity word $\omega$, the value
$T_\omega(x)$ coincides exactly with the result of applying the
Collatz map through the $k$ odd steps encoded by $\omega$, including
all intermediate divisions by two. Thus, the affine map $T_\omega$ represents
a compression of the corresponding finite Collatz trajectory into a single step.
\end{remark}

\begin{lemma}
\label{lem:residue_class}
Each convergent parity word $\omega$ induces a residue class
\[
x \equiv r_\omega \pmod{2^{A020914(k)}},
\]
whose elements all follow the same Collatz parity pattern for $k$ odd steps and reach
an integer strictly smaller than the starting value after at most $k$ odd steps.
\end{lemma}


\subsection{Increment Grammar and Horizontal-Sum Construction}

\begin{remark}[Increment Grammar from Minimal Denominator Growth]
The first-difference sequence of $A020914$, given by \(\OEIS{A022921}\),
induces a binary increment grammar \(\{s,d\}\) describing the growth of the
\emph{minimal admissible} total division count as the number of odd steps increases.

Specifically, each symbol records whether the minimal value of $A020914(k)$
increases by one (\(s\)) or by two (\(d\)) when passing from $k$ to $k+1$ odd steps.
This increment grammar does \emph{not} encode the realized parity word entries $d_i$, 
which may be arbitrarily large, but rather governs the evolution of the minimal denominator required for admissibility.

The horizontal-sum construction used to enumerate convergent parity words tracks
the accumulation of excess divisions above this minimal baseline, following the
algorithmic framework introduced by Winkler~\cite{Winkler2017}. Consequently,
the admissible local patterns in the $\{s,d\}$-grammar determine the combinatorial growth of $\widetilde A(k)$ independently of the specific values of the parity word entries.
\end{remark}

\begin{proposition}
The number of convergent parity words of length $k$ is given by \(\OEIS{A186009}\).
\end{proposition}

\begin{lemma}[Single-to-Double Row Sum Bound]
\label{lem:single-double-bound}
Let a row produced by a single step have total sum $k$. Apply a double step
immediately afterward. Then the resulting row has total sum at most $3k$.
\end{lemma}

\begin{proof}
By definition, the total sum of the single row equals $k$. Under a subsequent
double step, the terminal entry of the new row is formed as the sum of all
entries of the preceding row, and the previous terminal entry is duplicated.

Thus the double row consists of:
(i) the interior entries of the single row, whose total is $\le k$;
(ii) one terminal entry equal to $k$; and
(iii) a duplicated terminal entry, also equal to $k$.

Hence the total sum of the double row is bounded by
\[
k + k + k = 3k.
\]
\end{proof}

\begin{remark}
This lemma ensures that doubling steps contribute at most a controlled
multiplicative factor to horizontal-sum growth, independent of global cycle
structure. Consequently, all admissible local patterns within a $41$-cycle
admit finite maximal growth factors, computed explicitly by finite enumeration
in Appendix~\ref{app:local-growth-details}.
\end{remark}


\section{Cumulative Density of Convergent Classes}

\subsection{Density Contributions and Denominator Growth}

Define
\[
\widetilde A(i) := A186009(i+1),
\]
so that $\widetilde A(i)$ counts convergent residue classes of stopping time $i$.

Let
\[
B(i) := A020914(i),
\]
so that the density contribution of convergent residue classes of stopping time $i$ is
\[
\widetilde A(i)\, 2^{-B(i)}.
\]

By definition of $A020914$ as the minimal exponent with $2^{B(i)} > 3^i$, the function
$B(i)$ satisfies the following properties:
\begin{itemize}
    \item $B(i)$ is strictly increasing, with increments either 1 or 2 (never zero).
    \item Asymptotically, $B(i) \sim (\log_2 3)\, i$.
    \item By construction, $2^{B(i)} > 3^i$ for all $i \ge 0$.
\end{itemize}

Because $2^{B(i)}$ is a power of two and $B(i)$ is defined as the minimal
integer such that $2^{B(i)} > 3^i$, it follows that $2^{B(i)}$ is the smallest
power of two exceeding $3^i$. Since consecutive powers of two differ by a
factor of $2$, we obtain the sharp bound
\[
3^i < 2^{B(i)} < 2 \cdot 3^i.
\]

\begin{remark}[Indexing and Effective Growth Delay]
The OEIS sequences \(A020914\) and \(A186009\) employ differing indexing
conventions. In order to compare numerator and denominator growth on a common
scale, we define \(\widetilde A(i) := A186009(i+1)\), so that the index \(i\)
corresponds uniformly to stopping time (equivalently, parity-word length).

While the first doubling increment in \(A020914\) occurs at index \(1\), its
effect on \(\widetilde A\) appears after a fixed finite delay of two growth
steps. This delay is independent of \(i\) and therefore has no impact on
asymptotic growth-rate comparisons or density estimates.
\end{remark}

\noindent
\textbf{Constant-factor control.}
This bound depends only on the definition of $A020914$ and does not rely on
any delayed-doubling or numerator-growth arguments. Consequently, the reciprocal satisfies
\[
2^{-B(i)} \in \left( \frac{1}{2 \cdot 3^i}, \frac{1}{3^i} \right),
\]
giving a precise constant factor for the density contribution at level $i$.

We now record this observation formally for later reference.
\begin{lemma}[Uniform Control of Density Contributions]
\label{lem:uniform-control}
For all $i \ge 0$,
\[
3^i < 2^{A020914(i)} < 2\cdot 3^i.
\]
Consequently, the density contribution of convergent residue classes of
stopping time $i$ satisfies
\[
\frac{\widetilde A(i)}{2 \cdot 3^i}
\;<\;
\widetilde A(i)\,2^{-A020914(i)}
\;<\;
\frac{\widetilde A(i)}{3^i}.
\]
\end{lemma}

\begin{proof}
By definition, $A020914(i)$ is the minimal integer $m$ such that $2^m > 3^i$.
Consecutive powers of two differ by a factor of $2$, so
\[
3^i < 2^{A020914(i)} < 2 \cdot 3^i.
\]
Taking reciprocals and multiplying by $\widetilde A(i)$ gives the stated bounds.
\end{proof}


\subsection{Growth Rate of the Number of Convergent Classes}

\begin{theorem}[Maximal Admissible Growth Rate for $\widetilde A = A186009$]
\label{thm:maximal-growth}
Recall that $\widetilde A(i)=A186009(i+1)$ is generated by the horizontal-sum
construction associated with the increment grammar of \OEIS{A022921}, and let
$B(i):=A020914(i)$ denote the cumulative exponent of $2$ up to level $i$.

Then
\[
\limsup_{i\to\infty} \widetilde A(i)^{1/i} < 3,
\]
and consequently
\[
\frac{\widetilde A(i)}{2^{B(i)}} \longrightarrow 0 \quad \text{as } i\to\infty.
\]
\end{theorem}

\begin{proof}
The increment sequence \OEIS{A022921} induces a finite symbolic grammar on symbols
\[
\{ s, d \},
\]
corresponding respectively to single and double events in the horizontal-sum construction.
Every admissible infinite sequence of increments can be partitioned into concatenations
of two fundamental cycle types:
\begin{itemize}
  \item a \emph{7-cycle} with pattern
  \[
  \{ s, d, s, d, s, d, d \},
  \]
  and
  \item a \emph{5-cycle} with pattern
  \[
  \{ s, d, s, d, d \}.
  \]
\end{itemize}

Each cycle contributes a fixed multiplicative factor to the net growth. Explicitly,
\[
\frac{2^{11}}{3^{7}} < 1
\quad\text{for the 7-cycle,}
\qquad
\frac{2^{8}}{3^{5}} > 1
\quad\text{for the 5-cycle.}
\]
Thus 7-cycles are strictly contractive, while 5-cycles are expansive.

The minimal composite block that arises naturally in the grammar is the \emph{12-cycle},
formed by concatenation of a 7-cycle followed by a 5-cycle, with pattern
\[
\{ s, d, s, d, s, d, d, s, d, s, d, d \}.
\]
Its net multiplicative factor is
\[
\frac{2^{19}}{3^{12}} < 1,
\]
so the 12-cycle is also contractive.

Since 12-cycles are contractive, consecutive repetitions multiplicatively diverge from
unity until it becomes admissible to append a 5-cycle. The smallest admissible block
whose net effect exceeds unity is therefore the \emph{41-cycle}, consisting of three
consecutive 12-cycles followed by a 5-cycle. Its net multiplicative factor is
\[
\frac{2^{65}}{3^{41}} > 1.
\]

More generally, any admissible word can be parsed into concatenations of complete
7- and 5-cycles together with a finite prefix and/or suffix corresponding to a
partial cycle. Its total multiplicative growth is therefore the product of the
corresponding cycle factors multiplied by a bounded correction arising from these
initial and terminal fragments.

Since 7-cycles are contractive and 12-cycles remain contractive, any block whose
average multiplicative effect per symbol exceeds unity must contain at least one
5-cycle. However, the admissibility constraints imposed by the grammar force such
5-cycles to be separated by a minimum of three 12-cycles. Consequently, among all
admissible finite blocks, the quantity
\[
\bigl(\text{net growth factor of the block}\bigr)^{1/(\text{block length})}
\]
is maximized precisely by the minimal admissible arrangement achieving net expansion,
namely the 41-cycle.

Any admissible infinite word whose associated horizontal sums grow without bound
is therefore eventually dominated, in per-symbol growth rate, by sequences composed of
repeated 41-cycles or by longer blocks containing strictly more contractive content per
symbol. Since the 41-cycle maximizes per-symbol growth among all admissible patterns,
its infinite repetition provides a computable upper bound on $\widetilde A(i)$, and hence
on the maximal possible geometric growth rate of all admissible sequences.

The horizontal-sum dynamics within each local configuration combine Pascal-type summation
with delayed doubling effects. A complete extremal analysis over all admissible local
patterns occurring within a 41-cycle yields a maximal per-step growth factor
\[
\rho_{41} < 3.
\]
Since all other admissible infinite sequences contain strictly more contractive content,
their geometric growth rates are bounded above by $\rho_{41}$ as well. Therefore,
\[
\limsup_{i\to\infty} \widetilde A(i)^{1/i} \le \rho_{41} < 3.
\]

Finally, since the denominator satisfies
\[
3^i < 2^{B(i)} < 2\cdot 3^i,
\]
we obtain
\[
\frac{\widetilde A(i)}{2^{B(i)}} \le \frac{C\,\rho_{41}^i}{3^i} \longrightarrow 0,
\]
which completes the proof.
\end{proof}


\subsection{Vanishing of the Cumulative Density}

The preceding results show that although the number of convergent residue classes
$\widetilde A(i)$ grows with stopping time, its growth rate is strictly dominated
by the denominator $2^{B(i)}$, which scales asymptotically as $3^i$.

By Lemma~\ref{lem:uniform-control}, the density contribution at level $i$ satisfies
\[
\widetilde A(i)\,2^{-B(i)} \;\asymp\; \frac{\widetilde A(i)}{3^i}.
\]
By Theorem~\ref{thm:maximal-growth}, we have
\[
\frac{\widetilde A(i)}{3^i} \longrightarrow 0.
\]
Therefore the density contribution of newly appearing convergent residue classes
tends to zero as the stopping time increases.

In particular, the cumulative density of all convergent classes is dominated by a
summable sequence of vanishing contributions, ensuring that the set of integers
not belonging to any convergent residue class has asymptotic density zero.


\section{Global Descent}

\begin{theorem}[Global Descent]
Every positive integer eventually reaches a smaller positive integer
under iteration of the Collatz map.
\end{theorem}

\begin{proof}
By Theorem~\ref{thm:maximal-growth}, the density contribution of convergent
residue classes of stopping time $i$ satisfies
\[
\frac{\widetilde A(i)}{2^{B(i)}} \longrightarrow 0 \quad \text{as } i \to \infty.
\]
Consequently, the cumulative density of convergent residue classes approaches 1. 
Since the residue classes form a partition of the positive integers, 
every integer must belong to some convergent residue class of finite stopping time.

For each such residue class, there exists a finite integer $k$ such that
the $k$-fold application of the \emph{accelerated} Collatz map (counting only
odd steps, with all intervening divisions by $2$ absorbed into a single step)
satisfies
\[
f^k(x) < x.
\]
Since any additional divisions by $2$ required between odd steps do not increase
the number of odd iterations, the same inequality holds after a finite number of
iterations of the standard Collatz map. Hence every positive integer eventually
reaches a smaller positive integer under iteration of the Collatz map.
\end{proof}

Global Descent implies that no positive integer can remain perpetually above
its initial value, leaving only the trivial 1-2-4 cycle as a recurrent orbit.


\section{Non-Existence of Non-Trivial Cycles}

\begin{theorem}
The only cycle in the positive integers under the Collatz map is
\(4 \rightarrow 2 \rightarrow 1 \rightarrow 4\).
\end{theorem}


\section{Conclusion}

We have shown that the positive integers may be partitioned into
non-overlapping residue classes determined by finite Collatz parity
patterns, each assigned a minimal stopping time. By explicitly
enumerating convergent parity words and controlling the maximal admissible
growth rate of \(\widetilde A(i)\) relative to \(2^{B(i)}\), we proved that
the cumulative density of convergent classes converges to one.

This establishes a global descent mechanism for the accelerated Collatz
map: every positive integer eventually reaches a strictly smaller value,
ruling out the existence of non-trivial cycles. The argument is global
and structural in nature, relying neither on probabilistic heuristics
nor on bounds for individual trajectories.

The approach reframes the Collatz conjecture as an exhaustion problem
over arithmetic classes rather than a question of trajectory control,
suggesting a framework that may be applicable to related discrete
dynamical systems.


\newpage
\appendix
\section{Canonical Determination of the Constant \(c(\omega)\)}
\label{app:canonical-c}

In Lemma~\ref{lem:associated-map}, each parity word
\(\omega = (d_1,\dots,d_k)\) induces an affine Collatz map of the form
\[
T_\omega(x) = \frac{3^k x + c(\omega)}{2^{D}}, 
\qquad D = d_1 + \cdots + d_k.
\]
The constant \(c(\omega)\) is uniquely determined by the parity word
\(\omega\) up to congruence modulo \(2^D\). In this appendix, we describe
a canonical choice of representative and an explicit method for
computing it.

\subsection*{Existence and Uniqueness}

Fix a parity word \(\omega\) of length \(k\) and let \(D\) denote its
total division count. Any integer \(x\) that follows the parity pattern
\(\omega\) satisfies
\[
3^k x + c \equiv 0 \pmod{2^D}
\]
for some integer \(c\). Thus \(c\) is uniquely determined modulo \(2^D\),
and all admissible constants differ by integer multiples of \(2^D\).

To remove this ambiguity, we select the unique representative
\(c(\omega)\) satisfying
\[
0 \le c(\omega) < 2^D.
\]
This choice is canonical and depends only on the parity word \(\omega\).

\subsection*{Representative Independence}

The value \(c(\omega)\) is independent of the choice of integer
representative \(x\) within the residue class induced by \(\omega\).
Indeed, if \(x\) and \(x'\) belong to the same class modulo \(2^D\), then
\[
3^k x \equiv 3^k x' \pmod{2^D},
\]
yielding the same canonical value of \(c(\omega)\).

\subsection*{Explicit Computation (Step-by-Step)}

Given a parity word \(\omega\) of length \(k\) and total division count \(D\), 
the canonical constant \(c(\omega)\) may be computed as follows:

\begin{enumerate}
\item Choose any integer \(x\) belonging to the residue class induced by
      the parity word \(\omega\).
\item Compute \(y = 3^k x\).
\item Let \(m\) be the smallest integer greater than or equal to \(y / 2^D \).
\item Define
      \[
      c(\omega) = m \cdot 2^D - y.
      \]
\end{enumerate}

By construction, \(0 \le c(\omega) < 2^D\), and the resulting value
satisfies
\[
T_\omega(x) = \frac{3^k x + c(\omega)}{2^D} \in \mathbb{Z}.
\]
This value is independent of the chosen representative \(x\) and
provides a concrete realization of the affine map associated with \(\omega\).

\subsection*{Concrete Example}

Consider the parity word
\(\omega = (1,1,1,4)\) with \(k=4\) and total division count
\(D = 1 + 1 + 1 + 4 = 7\).

\begin{enumerate}
\item Choose a representative \(x = 15\) in the corresponding residue class.
\item Compute \(y = 3^4 \cdot 15 = 81 \cdot 15 = 1215\).
\item Find the smallest \(m\) such that \(m \cdot 2^7 \ge 1215\):
      \(2^7 = 128\), so \(m = \lceil 1215/128 \rceil = 10\).
\item Compute \(c(\omega) = 10 \cdot 128 - 1215 = 1280 - 1215 = 65\).
\end{enumerate}

Thus, the canonical affine map for this parity word is
\[
T_\omega(x) = \frac{3^4 x + 65}{2^7} = \frac{81x + 65}{128}.
\]

Checking another representative \(x = 143\) in the same residue class:
\[
T_\omega(143) = \frac{81 \cdot 143 + 65}{128} = \frac{11583 + 65}{128} = 91,
\]
confirming that the same \(c(\omega)\) works uniformly across the residue class.


\newpage
\section{Extremal Local Growth Analysis}

\begin{lemma}[Local Growth Bounds]
\label{lem:local-growth-bounds}
For each admissible local pattern occurring within a $41$-cycle,
the ratio of the total sum of entries after applying the pattern
to the total sum before applying it is bounded above as follows,
uniformly over all admissible placements of delayed doubling within
the pattern:
\[
\begin{array}{c|c}
\text{Pattern} & \text{Max Growth Factor} \\
\hline
s,d & 3 \\
d,d & 123/33 \\
s,d,s & 341/131 \\
d,d,s & 329/123
\end{array}
\]
\end{lemma}

\begin{proposition}[Extremal 41-Cycle Growth]
The maximal geometric growth rate achievable by any admissible infinite
word under the horizontal-sum dynamics ~\cite{Winkler2017} is bounded by
\[
\rho_{41}
=
\Bigl(
3^{17}
\cdot (123/33)^7
\cdot (341/131)^{10}
\cdot (329/123)^7
\Bigr)^{1/41}
< 3,
\]
where the exponents count the occurrences of each local pattern within a
$41$-cycle.
\end{proposition}

\noindent
Here the exponents record the total number of coefficients generated by each
admissible local pattern within a single $41$--cycle; the enumeration of these
patterns and the derivation of the counts $17,7,10,7$ are given explicitly in
Appendix~\ref{app:local-growth-details}, Section~\ref{sec:admissible-patterns}.

\begin{remark}[Derivation of Local Growth Constants]
The constants shown in Lemma~\ref{lem:local-growth-bounds} are obtained
by a complete extremal analysis of the horizontal-sum dynamics governing
$\widetilde A(i)$.  Within each local configuration, elements arising from
single steps evolve according to standard Pascal-type summation, yielding
maximal growth proportional to powers of $2$, while additional entries
introduced by doubling steps are bounded by summing on top of this baseline
growth.

For each admissible local pattern occurring within a $41$-cycle, all
possible placements of delayed doubling contributions were examined, and the
largest achievable ratio of post-pattern to pre-pattern row sums was
recorded.  No assumptions beyond the combinatorial structure of the
horizontal-sum construction are used.

Full illustrative computations are provided in Appendix~\ref{app:local-growth-details}.
\end{remark}


\newpage
\section{Extremal Local Growth Calculations}
\label{app:local-growth-details}

Recall that the sequence \OEIS{A020914} has first differences given by
\OEIS{A022921}.  This increment sequence consists of repeated blocks
composed of \emph{single} steps ($s$) and \emph{double} steps ($d$).

\subsection{Cycle Structure and Net Multiplicative Effect}

A frequently occurring block is the $7$--cycle
\[
\{ s, d, s, d, s, d, d \},
\]
which contains $3$ singles and $4$ doubles.  The corresponding number of
factors of $2$ is
\[
3 + 2(4) = 11,
\]
so the net multiplicative contribution of this cycle satisfies
\[
\frac{2^{11}}{3^7} < 1.
\]
Thus the $7$--cycle is strictly contractive.

This $7$--cycle is always followed by a $5$--cycle
\[
\{ s, d, s, d, d \},
\]
which contains $2$ singles and $3$ doubles, contributing
\[
2 + 2(3) = 8
\]
factors of $2$.  Since
\[
\frac{2^8}{3^5} > 1,
\]
the $5$--cycle is expansive.

Concatenating the $7$--cycle and $5$--cycle produces a $12$--cycle
\[
\{ s, d, s, d, s, d, d, s, d, s, d, d \},
\]
containing $5$ singles and $7$ doubles.  The total number of factors of $2$ is
\[
5 + 2(7) = 19,
\]
and hence
\[
\frac{2^{19}}{3^{12}} < 1.
\]
Therefore, the $12$--cycle is again contractive.

Repeated $12$--cycles diverge multiplicatively from unity until it becomes
possible to append an additional $5$--cycle, producing a $41$--cycle composed
of three consecutive $12$--cycles followed by a $5$--cycle.  The total number
of factors of $2$ in a $41$--cycle is
\[
3(19) + 8 = 65,
\]
and since
\[
\frac{2^{65}}{3^{41}} > 1,
\]
the $41$--cycle is expansive.

Importantly, a sequence composed entirely of consecutive $41$--cycles
strictly dominates the growth of \OEIS{A186009}, which also admits longer
$53$--cycles (four $12$--cycles followed by a $5$--cycle).  Consequently, any
asymptotic growth bound obtained using only $41$--cycles provides an upper
bound for the growth rate of \OEIS{A186009} itself.

\subsection*{Horizontal-Sum Dynamics and Pascal-Type Growth}

Due to doubling events, the usual $2\times$ growth limitation of a standard
Pascal triangle does not apply directly.  In particular, the sum of earlier
row entries may exceed the final entry of the row.  However, single steps do
not increase the length of the generating tuple.  As a result, growth
starting from singles follows standard Pascal-type summation rules, while
doubling events introduce additional entries whose contributions must be
handled separately.

For any single step, the generating terms are strictly increasing from left
to right across the row.  The maximal growth therefore arises from combining
Pascal-type summation and doubling events as per Lemma~\ref{lem:single-double-bound}.

\newpage
\subsection*{Admissible Local Patterns}
\label{sec:admissible-patterns}

Within a $41$--cycle, only the following four local configurations can occur.
This is a consequence of the cycle structure described above.  Each $41$--cycle
is composed of repeated $12$--cycles and a terminal $5$--cycle, where the only
step patterns are
\[
\{ s, d, s, d, s, d, d, s, d, s, d, d \} \quad \text{and} \quad \{ s, d,s,d,d \}.
\]
In particular, no two single steps can occur consecutively, while double steps
may occur either singly or in pairs.  Examining all contiguous subwords of these
cycles therefore yields exactly four admissible local patterns:
\[
(s, d),\quad (d, d),\quad (s, d, s),\quad (d, d, s),
\]
and no others.

Enumerating the number of occurrences of each pattern within a $41$--cycle gives
the following counts:
\[
\begin{array}{c|c}
\text{Pattern} & \text{Associated coefficient count} \\
\hline
s,d & 3(5) + 2 = 17 \\
d,d & 3(2) + 1 = 7 \\
s,d,s & 3(3) + 1 = 10 \\
d,d,s & 3(2) + 1 = 7
\end{array}
\]

\subsection{Illustrative Extremal Computations}

The generating rows evolve by a horizontal--sum rule:
\[
a_{n+1}(i+1) \;=\; a_{n+1}(i) + a_n(i+1),
\]
so that sequences of single steps exhibit ordinary Pascal--type growth.
Double steps differ only in that the terminal entry of the row is duplicated,
causing the final term to appear twice in the generating tuple while preserving
monotonicity of the preceding entries.

The following examples are not special cases but representative extremal probes
of the same underlying recurrence.  Since the horizontal--sum recurrence and
Lemma~\ref{lem:single-double-bound} provide a uniform inequality valid for all
admissible generating rows, any concrete realization that saturates these
inequalities yields an upper bound that applies to every occurrence of the same
local pattern.

We illustrate the bounding procedure with concrete examples.  Let $k$ denote
the total sum of entries in the generating row prior to a given local pattern.
By Lemma~\ref{lem:single-double-bound}, each single-to-double transition is
bounded which provides the basis for the extremal analysis:

\paragraph{Example 1.}
\[
\begin{array}{l}
\text{Single } a(10) = \underbrace{173}_{\text{sum} := k}:
\underbrace{1,7,25,55,85}_{k}
\end{array}
\]

\[
\begin{array}{l}
\text{Double } a(11) = \underbrace{476}_{\text{sum} = 3k}:
\underbrace{1,8,33,88}_{k}
\underbrace{173}_{k},
\underbrace{173}_{k}
\end{array}
\]

\[
\begin{array}{l}
\text{Single } a(12) = \underbrace{961}_{\text{sum} = 9k}:
\underbrace{1,9,42,130}_{2k}
\underbrace{303}_{3k},
\underbrace{476}_{4k}
\end{array}
\]

\[
\begin{array}{l}
\text{Double } a(13) = \underbrace{2652}_{\text{sum} = 33k}:
\underbrace{1,10,52,182}_{4k},
\underbrace{485}_{7k},
\underbrace{961}_{11k},
\underbrace{961}_{11k}
\end{array}
\]

\[
\begin{array}{l}
\text{Double } a(14) = \underbrace{8045}_{\text{sum} = 123k}:
\underbrace{1,11,63,245}_{8k},
\underbrace{730}_{15k},
\underbrace{1691}_{26k},
\underbrace{2652}_{37k},
\underbrace{2652}_{37k}
\end{array}
\]

\[
\begin{array}{l}
\text{Single } a(15) = \underbrace{17637}_{\text{sum} = 329k}:
\underbrace{1,12,75,320}_{16k},
\underbrace{1050}_{31k},
\underbrace{2741}_{57k},
\underbrace{5393}_{94k},
\underbrace{8045}_{131k}
\end{array}
\]

From these bounds, the maximal local ratios are
\[
\begin{array}{c|c}
\text{Pattern} & \text{Max Ratio} \\
\hline
s,d & 3 \\
d,d & 123/33 \\
s,d,s & 3 \\
d,d,s & 329/123
\end{array}
\]

\newpage
\paragraph{Example 2.}
\[
\begin{array}{l}
\text{Single } a(12) = \underbrace{961}_{\text{sum} := k}:
\underbrace{1,9,42,130,303,476}_{k}
\end{array}
\]

\[
\begin{array}{l}
\text{Double } a(13) = \underbrace{2652}_{\text{sum} = 3k}:
\underbrace{1,10,52,182,485}_{k},
\underbrace{961}_{k},
\underbrace{961}_{k}
\end{array}
\]

\[
\begin{array}{l}
\text{Double } a(14) = \underbrace{8045}_{\text{sum} = 13k}:
\underbrace{1,11,63,245,730}_{2k},
\underbrace{1691}_{3k},
\underbrace{2652}_{4k},
\underbrace{2652}_{4k}
\end{array}
\]

\[
\begin{array}{l}
\text{Single } a(15) = \underbrace{17637}_{\text{sum} = 37k}:
\underbrace{1,12,75,320,1050}_{4k},
\underbrace{2741}_{7k},
\underbrace{5393}_{11k},
\underbrace{8045}_{15k}
\end{array}
\]

\[
\begin{array}{l}
\text{Double } a(16) = \underbrace{51033}_{\text{sum} = 131k}:
\underbrace{1,13,88,408,1458}_{8k},
\underbrace{4199}_{15k},
\underbrace{9592}_{26k},
\underbrace{17637}_{41k},
\underbrace{17637}_{41k}
\end{array}
\]

\[
\begin{array}{l}
\text{Single } a(17) = \underbrace{108950}_{\text{sum} = 341k}:
\underbrace{1,14,102,510,1968}_{16k},
\underbrace{6167}_{31k},
\underbrace{15759}_{57k},
\underbrace{33396}_{98k},
\underbrace{51033}_{139k}\end{array}
\]

From these bounds, the maximal local ratios are
\[
\begin{array}{c|c}
\text{Pattern} & \text{Max Ratio} \\
\hline
s,d & 3 \\
d,d & 13/3 \\
s,d,s & 341/131 \\
d,d,s & 37/13
\end{array}
\]

\subsection*{Improved Extremal Alignment}

Because the total multiplicative growth over a cycle is the product of the local
growth factors, the geometric mean over one full cycle is maximized by selecting
the maximal admissible factor at each local pattern occurrence. Consequently,
once a uniform upper bound has been established for each admissible local pattern,
the extremal global growth rate over a complete cycle is obtained by multiplying
these maximal local factors according to their frequencies and taking the
corresponding geometric mean.

Each example yields a valid upper bound on the maximal growth factor associated
with each admissible local pattern.  Since these bounds arise from the same
uniform recurrence and apply globally to all admissible configurations, the
true maximal growth factor for a given pattern is bounded above by the minimum
of all such valid upper bounds.

Since the recurrence is monotone with respect to the placement of delayed doubling,
no other admissible configuration can exceed these extremal realizations.
Accordingly, combining the sharpest bounds obtained across different initiating
rows yields the tightest global envelope on admissible local growth.
\[
\begin{array}{c|c|c}
\text{Pattern} & \text{Max Ratio} & \text{Associated coefficient count} \\
\hline
s,d & 3 & 17 \\
d,d & 123/33 & 7 \\
s,d,s & 341/131 & 10 \\
d,d,s & 329/123 & 7
\end{array}
\]

Consequently, the maximal geometric growth rate of a $41$--cycle satisfies
\[
\Bigl(
3^{17}
\cdot (123/33)^7
\cdot (341/131)^{10}
\cdot (329/123)^7
\Bigr)^{1/41}
< 2.95 < 3.
\]

This bound applies \emph{a fortiori} to \OEIS{A186009}, whose admissible growth
patterns are strictly dominated, in the sense of per--symbol multiplicative
growth, by those of a pure $41$--cycle concatenation.

\paragraph{Conclusion.}
Since every admissible growth pattern of \OEIS{A186009} is composed of
$12$--cycles and $5$--cycles, and since concatenations of $41$--cycles
maximize per--symbol multiplicative growth among all such admissible
concatenations, the bound obtained above for the $41$--cycle applies
\emph{a fortiori} to \OEIS{A186009} itself.


\newpage
\section{Parity-Word Growth and the ``Less Than Unity'' Invariant}
\label{app:less-than-unity}

The OEIS sequences $A020914$ and $A022921$ encode the minimal and maximal
growth of admissible parity words for the accelerated Collatz map.  In
this appendix we reformulate their defining property as a multiplicative
\emph{less-than-unity invariant}, which makes explicit the contraction
mechanism governing admissible trajectories and yields immediate upper
bounds on all growth sequences, including $A186009$.

\subsection{Step Contributions and Cumulative Contraction}

Fix a parity word $\omega=(d_0,\dots,d_k)$ of length $k+1$, where each
$d_i\in\{1,2\}$ represents the number of divisions by $2$ following the
$i$th odd step.  Define the normalized step contribution
\[
r_i := \frac{2^{d_i}}{3}.
\]
Individual factors $r_i$ may be greater or less than unity; admissibility
is determined by the cumulative product.

\begin{quote}
We interpret each step as contributing a factor $r_i$, choosing the
\emph{largest possible $d_i$ such that the cumulative product remains
strictly less than unity}, i.e.
\[
\prod_{i=0}^{k} r_i
=
\prod_{i=0}^{k} \frac{2^{d_i}}{3}
< 1.
\]
\end{quote}

Equivalently,
\begin{align*}
\prod_{i=0}^{k} \frac{2^{d_i}}{3} < 1
&\iff
\frac{2^{\sum_{i=0}^{k} d_i}}{3^{k+1}} < 1,\\
&\iff
2^{\sum_{i=0}^{k} d_i} < 3^{k+1}.
\end{align*}
Thus admissibility is characterized by the requirement that the total
power of $2$ accumulated along the parity word remain strictly below the
next power of $3$.

\subsection{Relation to \texorpdfstring{$A020914$}{A020914} and \texorpdfstring{$A022921$}{A022921}}

By definition, $A020914(k)$ is the minimal exponent $m$ such that
\[
2^{m} > 3^{k}.
\]
Equivalently, it is the smallest power of $2$ exceeding $3^{k}$, so that
\[
3^{k} < 2^{A020914(k)} < 3^{k+1}.
\]

The first-difference sequence $A022921$ records the incremental growth of
$A020914$.  Let
\[
S(k) := \sum_{i=0}^{k} A022921(i).
\]
Then $S(k)$ is the \emph{largest} exponent such that
\[
2^{S(k)} < 3^{k+1},
\]
i.e.\ the greatest power of $2$ strictly below $3^{k+1}$.  Consequently,
\[
S(k) \in \{A020914(k),\, A020914(k)+1\},
\]
reflecting the fact that $A020914$ gives the minimal exponent exceeding
$3^{k}$, while the cumulative sum of $A022921$ gives the maximal exponent
remaining below $3^{k+1}$.

Therefore, for a parity word with increments $d_0,\dots,d_k$,
\[
\sum_{i=0}^{k} d_i = S(k)
\quad\Longleftrightarrow\quad
2^{\sum_{i=0}^{k} d_i} < 3^{k+1},
\]
which is exactly the less-than-unity condition.

\subsection{Unity as a Strict Admissibility Bound}

The multiplicative formulation provides a sharper admissibility test than
the OEIS definitions alone.

\begin{lemma}[Unity Bound]
\label{lem:unity-bound}
If
\[
\prod_{i=0}^{k} \frac{2^{d_i}}{3} \ge 1,
\]
then
\[
\sum_{i=0}^{k} d_i \ge S(k)+1 \ge A020914(k)+1.
\]
\end{lemma}

\begin{proof}
The hypothesis is equivalent to $2^{\sum_{i=0}^{k} d_i} \ge 3^{k+1}$.  Since
$S(k)$ is defined as the largest exponent satisfying
$2^{S(k)} < 3^{k+1}$, any exponent meeting or exceeding $3^{k+1}$ must
satisfy $\sum_{i=0}^{k} d_i \ge S(k)+1$.  Because
$S(k)\ge A020914(k)$, the stated inequality follows.
\end{proof}

\begin{remark}[Prefix Invariance of Admissibility]
\label{rem:prefix-invariant}
Admissibility is not merely a terminal condition on a parity word but a
\emph{pathwise} constraint.  For a parity word $\omega=(d_0,\dots,d_k)$,
define the partial products
\[
P_j := \prod_{i=0}^{j} \frac{2^{d_i}}{3}, \qquad 0\le j\le k.
\]
Then $\omega$ is admissible if and only if
\[
P_j < 1 \quad \text{for all } j=0,1,\dots,k.
\]
In particular, if $P_j\ge 1$ for any prefix of the word, then the sequence
already exceeds the maximal admissible growth encoded by $A022921$ and
hence also the minimal boundary $A020914$.  Such a word is therefore
inadmissible regardless of any subsequent iterations, which for example
explains why $1,1,1,4$ is admissible while $4,1,1,1$ is not.
\end{remark}

Thus, crossing unity implies growth exceeding not only the minimal
boundary $A020914(k)$ but also the maximal admissible growth encoded by
$A022921$.  In this sense, the unity condition is a \emph{strict
invariant}: any parity word whose cumulative product reaches or exceeds
unity cannot be admissible.

\subsection{Consequences for Extremal Growth}

This invariant provides an immediate interpretation of extremal
configurations.  Any admissible sequence must satisfy
\[
\prod_{i=0}^{k} \frac{2^{d_i}}{3} < 1,
\]
while any pattern whose cumulative product equals or exceeds unity
constitutes a hard upper bound on admissible growth.  Consequently:

\begin{itemize}
  \item $A020914$ describes the minimal exponent at which expansion beyond
        $3^{k}$ becomes possible.
  \item The cumulative sum of $A022921$ gives the maximal exponent still
        remaining below $3^{k+1}$.
  \item The unity test identifies, in multiplicative terms, the exact
        boundary between admissible (contractive) and inadmissible
        (expansive) behavior.
\end{itemize}

In particular, any explicit parity-word pattern whose cumulative factor
exceeds unity necessarily dominates both $A020914$ and $A022921$ and
therefore provides an absolute upper bound on all admissible growth
trajectories.  This principle underlies the extremal cycle analysis in
Appendices~B and~C and yields corresponding upper bounds for the
horizontal-sum sequence $\widetilde A(i)=A186009(i+1)$.

\subsection{Summary}

The ``less than unity'' perspective is not a reinterpretation of the OEIS
definitions but an explicit invariant formulation of them.  It unifies:

\begin{itemize}
  \item the minimal growth boundary encoded by $A020914$,
  \item the maximal admissible growth encoded by the cumulative sums of
        $A022921$, and
  \item the multiplicative contraction condition governing admissible
        parity words.
\end{itemize}

Any parity word whose cumulative product exceeds unity necessarily
violates admissibility and hence forms a natural upper bound on all
growth sequences derived from the accelerated Collatz dynamics,
including $A186009$.


\newpage
\section{Extremal Cycle Structure and the Emergence of the 41-Cycle}
\label{app:extremal-cycles}

This appendix completes the structural analysis initiated in
Appendix~\ref{app:less-than-unity}.  There we established the
\emph{less-than-unity invariant} for admissible parity words:
for a sequence of division counts $\omega=(d_0,\dots,d_{k-1})$,
admissibility requires that every prefix satisfy
\[
\prod_{i=0}^{j} \frac{2^{d_i}}{3} < 1
\qquad \text{for all } 0\le j<k.
\]
Equivalently, no partial trajectory may ever exceed unity in cumulative
growth.  This invariant is strictly stronger than the endpoint condition
and provides a combinatorial grammar for admissible growth.

In this appendix we use that invariant to classify the admissible
composite patterns of increments induced by $A022921$ and to show that:
\begin{itemize}
  \item the only fundamental building blocks are the contractive
        7-cycle and the expansive 5-cycle;
  \item these combine into a contractive 12-cycle;
  \item shorter expansive composites (such as the 29-cycle) are
        inadmissible;
  \item the first \emph{minimal expansive} configuration compatible with
        the invariant is the 41-cycle; and
  \item repeated 53-cycles provide the contractive buffering that allows
        the 41-cycle to appear without violating admissibility.
\end{itemize}
Together these facts explain why the 41-cycle governs the extremal growth
of $A020914$, $A022921$, and hence the horizontal-sum sequence $A186009$.

\subsection{Increment Grammar from the Unity Invariant}

As in Appendix~\ref{app:less-than-unity}, let $d_i\in\{1,2\}$ denote the
increment at step $i$ (single or double), and write
\[
r_i=\frac{2^{d_i}}{3}.
\]
Although an individual factor $r_i$ may exceed unity, admissibility
requires that all partial products remain strictly below unity.

Direct evaluation of initial admissible prefixes yields the basic
grammar:
\begin{align*}
s &= \frac{2^1}{3} < 1, &d &= \frac{2^2}{3} > 1 \quad \text{(forbidden)},\\
s,s &= \frac{2^2}{3^2} < 1, &s,d &= \frac{2^3}{3^2} < 1 \quad \text{(maximal)},\\
s,d,s &= \frac{2^4}{3^3} < 1, &s,d,d &= \frac{2^5}{3^3} > 1 \quad \text{(forbidden)},\\
s,d,s,s &= \frac{2^5}{3^4} < 1, &s,d,s,d &= \frac{2^6}{3^4} < 1 \quad \text{(maximal)},\\
s,d,s,d,s &= \frac{2^7}{3^5} < 1, &s,d,s,d,d &= \frac{2^8}{3^5} > 1 \quad \text{(forbidden)},\\
s,d,s,d,s,s &= \frac{2^8}{3^6} < 1, &s,d,s,d,s,d &= \frac{2^9}{3^6} < 1 \quad \text{(maximal)},\\
s,d,s,d,s,d,s &= \frac{2^{10}}{3^7} < 1, &s,d,s,d,s,d,d &= \frac{2^{11}}{3^7} < 1 \quad \text{(maximal)}.
\end{align*}

This produces the maximal 7-cycle composed of 
\[
\{s,d,s,d,s,d,d\}.
\]

Thus:
\begin{itemize}
  \item every $s$ must be followed by a $d$;
  \item a $d$ may be followed by either $s$ or $d$;
  \item no prefix may violate the unity bound.
\end{itemize}
This symbolic grammar is equivalent to the first-difference structure of
$A022921$ and underlies the admissible configurations of division counts.

\subsection{The 5-Cycle and the Contractive 12-Cycle}

The next consecutive double ($d,d$) event occurs only after a full 7-cycle,
yielding the 5-cycle pattern 
\[
\{s,d,s,d,d\}
\]
as follows (suppressing forbidden and non-maximal patterns):
\begin{align*}
s,d,s,d,s,d,d,s &= \frac{2^{12}}{3^8} < 1,\\
s,d,s,d,s,d,d,s,d &= \frac{2^{14}}{3^9} < 1,\\
s,d,s,d,s,d,d,s,d,s &= \frac{2^{15}}{3^{10}} < 1,\\
s,d,s,d,s,d,d,s,d,s,d &= \frac{2^{17}}{3^{11}} < 1,\\
s,d,s,d,s,d,d,s,d,s,d,d &= \frac{2^{19}}{3^{12}} < 1.
\end{align*}

Concatenating a 7-cycle followed by a 5-cycle yields the 12-cycle
\[
\{s,d,s,d,s,d,d,s,d,s,d,d\},\qquad
\frac{2^{19}}{3^{12}} < 1.
\]
This 12-cycle is contractive, and repetition drives the cumulative
product further below unity.  

\subsection{Why Shorter Expansive Blocks Are Forbidden}

Consider a composite of two 12-cycles followed by a 5-cycle (a 29-cycle):
\[
\text{29-cycle} = 12 + 12 + 5,
\]
with net factor
\[
\frac{2^{46}}{3^{29}} > 1.
\]
The prefix-wise invariant forbids any prefix from exceeding unity; hence
the 29-cycle is inadmissible as an initial block and cannot be appended
without sufficient prior contraction.  

\subsection{Contractive Buffers: The 53-Cycle}

Longer composites can serve as contractive buffers. For instance,
\[
\text{53-cycle} = 12 + 12 + 12 + 12 + 5,
\]
has net factor
\[
\frac{2^{84}}{3^{53}} \approx 0.9979 < 1.
\]
Repetition of 53-cycles
\[
\left(\frac{2^{84}}{3^{53}}\right)^n \longrightarrow 0
\]
drives the cumulative product farther below unity, creating \emph{slack} in
which an expansive configuration may later be inserted without violating
the prefix-wise invariant.  This slack is essential for permitting the
first minimal expansive insertion.

\subsection{The Minimal Expansive Configuration: The 41-Cycle}

The smallest expansive pattern that can be inserted after sufficient
buffering is the 41-cycle,
\[
\text{41-cycle} = 12 + 12 + 12 + 5,
\]
with net factor
\[
\frac{2^{65}}{3^{41}} > 1.
\]

Its admissibility depends on prior contraction:
\begin{itemize}
  \item repeated 53-cycles drive the cumulative product below unity;
  \item after sufficient buffering, the 41-cycle can be appended without
        any prefix exceeding unity;
  \item no shorter expansive block (such as the 29-cycle) can be
        accommodated.
\end{itemize}
Hence the 41-cycle is the minimal expansive extremal under the unity invariant.

\subsection{Consequences for Extremal Growth}

The analysis establishes that:
\begin{enumerate}
  \item all admissible sequences are built from 7- and 5-cycles, with
        contractive 12-cycles as the dominant neutral unit;
  \item contractive 53-cycles serve as buffers, creating arbitrarily large
        slack in the prefix-wise unity invariant;
  \item the 41-cycle is the minimal expansive configuration that can be
        inserted after such buffering, maximizing admissible per-symbol growth.
\end{enumerate}

Thus any admissible infinite word whose horizontal sums grow without bound
is eventually dominated, in average growth per symbol, by repetitions of
the 41-cycle or by blocks with strictly more contractive content.  
This establishes the 41-cycle as the extremal configuration governing the
maximal growth of $A020914$ and $A022921$, and therefore bounds the
asymptotic growth of the horizontal-sum sequence
$\widetilde A(i)=A186009(i+1)$.

\begin{remark}[Structural Summary]
Appendix~\ref{app:less-than-unity} introduced the prefix-wise
less-than-unity invariant.  The present appendix shows how that invariant
forces a rigid combinatorial grammar: admissible words are built from
7- and 5-cycles, organize into contractive 12-cycles, require 53-cycles as
buffers, and admit the 41-cycle as the minimal expansive extremal.
Together, these facts provide a complete structural explanation for the
observed bounds on denominator growth and on the number of convergent
classes.
\end{remark}


\newpage
\bibliographystyle{unsrtnat}
\bibliography{src/latex/references}

\end{document}  
