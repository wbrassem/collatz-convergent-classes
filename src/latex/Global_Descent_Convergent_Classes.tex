\documentclass[12pt, oneside]{article}   	% use "amsart" instead of "article" for AMSLaTeX format
%\documentclass[12pt, oneside, draft]{article}   	% use "amsart" instead of "article" for AMSLaTeX format

%%%%%%%%%%%%%%%%%%%%%%%%%%%%%%%%%%%%%%%%%%%%%%%%%
%
% The TexLive distribution is stored in /usr/local/texlive/2020
%
% The definitions of the packages are in /usr/local/texlive/2020/texmf-dist/tex/latex
%
%%%%%%%%%%%%%%%%%%%%%%%%%%%%%%%%%%%%%%%%%%%%%%%%%

\usepackage{xparse}					% Allows for the creation of \NewDocumentCommand shortcuts

\usepackage{geometry}				% See geometry.pdf to learn the layout options. There are lots.
\geometry{letterpaper}				% ... or a4paper or a5paper or ... 
\usepackage[parfill]{parskip}		% Activate to begin paragraphs with an empty line rather than indent
\usepackage{float}					% Float package to exert more control over graphic placement
	
%\usepackage{verbatim}				% Creates a verbatim environment for code / program output
\usepackage{graphicx}				% Use pdf, png, jpg, or eps§ with pdflatex; use eps in DVI mode
									% TeX will automatically convert eps --> pdf in pdflatex
\graphicspath{{./figures/}}         

% Math packages
\usepackage{amssymb}				% Mathematical symbols
\usepackage{amsmath}				% Added for typesetting mathematical formulae
\usepackage{amsthm}					% Added for typesetting mathematical theorems
\usepackage{mathalpha}				% Added for some special characters like Q, Z and N

% Table packages in addition to built in tabular
\usepackage{booktabs}				% Adds extra commands to tabular like \toprule
\usepackage{afterpage}				% For pagination support
\usepackage{longtable}				% Support for longer tables

% Hyperlinks
\usepackage{hyperref}				% For creating internal and HTTP links

% For Tikz graphics
\usepackage{tikz}					% Package for pgf and TikZ
\usetikzlibrary{shapes,arrows,chains}
\tikzstyle{startstop} = [very thick, rectangle, rounded corners, minimum width=2.5cm, minimum height=0.8cm, text centered, draw=black, text width=2.5cm]
\tikzstyle{action} = [rectangle, rounded corners, minimum width=3cm, minimum height=0.8cm, text centered, draw=black, text width=2.5cm]
\tikzstyle{decision} = [diamond, minimum width=3cm, minimum height=3cm, text centered, draw=black, text width=2cm]
\tikzstyle{arrow} = [thick,->,>=stealth]

% For theorems and definitions - each creates it's own counter which incrementa each use
\theoremstyle{definition}
\newtheorem{proposition}{Proposition}[section]
\newtheorem{define}{Definition}[section]
\newtheorem{axiom}{Axiom}[section]
\newtheorem{theorem}{Theorem}
\newtheorem{lemma}[theorem]{Lemma}
\newtheorem{corollary}[theorem]{Corollary}
\newtheorem*{remark}{Remark}

%SetFonts


\title{Global Descent and Convergent Classes in the Collatz Conjecture}
\author{Wayne Brassem}
%\date{}							% Activate to display a given date or no date


\begin{document}

% Document define commands for commonly used items in math mode
\NewDocumentCommand{\setN}{}{\mathbb{N}}				% Set of positive integers, excluding 0
\NewDocumentCommand{\setNo}{}{\mathbb{N}_0}				% Set of positive integers, including 0
\NewDocumentCommand{\setNeven}{}{\mathbb{N}_{even}}		% Set of even positive integers, excluding 0
\NewDocumentCommand{\setNodd}{}{\mathbb{N}_{odd}}		% Set of odd positive integers
\NewDocumentCommand{\setZ}{}{\mathbb{Z}}				% Set of all integers, excluding zero
\NewDocumentCommand{\setZo}{}{\mathbb{Z}_0}				% Set of all integers, including zero
\NewDocumentCommand{\setQ}{}{\mathbb{Q}}				% Set of all rational numbers

% Document define commands for commonly used items not in math mode
\NewDocumentCommand{\Rarr}{}{\textrightarrow{}}			% Right arrow

% Document commands which provide hyperlinks to OEIS sequences referenced herein
\NewDocumentCommand{\OEIS}{m}{\href{https://oeis.org/#1}{#1}}

\maketitle



\begin{abstract}

%Abstract - goals of the paper
We study the accelerated Collatz map

\[
f(x) = \begin{cases}
\displaystyle
    \phantom{3x} \frac{x}{2} & \text {if } x \text{ is even}, \\[8pt]  % Insert some spacing between rows
\displaystyle
    \frac{3x+1}{2}           & \text {if } x \text{ is odd}. \\
\end{cases}
\]

and analyze the structure of integer trajectories through parity-encoded residue classes. Each finite
parity word induces a non-overlapping residue class modulo a power of two, and convergent parity words
correspond to classes whose elements reach a strictly smaller integer after finitely many iterations.

Using explicit enumeration results for parity words and their associated division counts, we show that
convergent residue classes admit well-defined asymptotic densities. These classes may be grouped by
minimal stopping time without overlap, yielding a cumulative density that increases monotonically with
stopping time. We prove that the complement of this union has asymptotic density tending to zero,
implying that the cumulative density of convergent classes converges to one.

As a consequence, every positive integer belongs to a convergent residue class of finite stopping time
and therefore eventually reaches a smaller positive integer under iteration of the Collatz map. This
establishes global descent and rules out the existence of nontrivial cycles in the positive integers,
leaving the classical 4\Rarr2\Rarr1\Rarr4 cycle as the unique periodic orbit.

\end{abstract}
\newpage


\tableofcontents
\newpage


\section{Introduction}

\subsection{The Collatz Mapping and Known Results}

The Collatz conjecture concerns the behavior of the (accelerated) map

\[
f(x) = \begin{cases}
\displaystyle
    \phantom{3x} \frac{x}{2} & \text {if } x \text{ is even}, \\[8pt]  % Insert some spacing between rows
\displaystyle
    \frac{3x+1}{2}           & \text {if } x \text{ is odd}. \\
\end{cases}
\]

acting on the positive integers. The conjecture asserts that, under
iteration of \(f\), every positive integer eventually reaches the
cycle \(4 \to 2 \to 1 \to 4\).

Despite its elementary formulation, the conjecture has resisted proof
for decades. Extensive computational verification confirms convergence
for all integers up to very large bounds, but such verification does not
constitute a proof and offers limited insight into the global structure
of Collatz trajectories.

Many prior approaches focus on bounding individual trajectories or
analyzing probabilistic heuristics. While these methods provide valuable
intuition, they do not address the problem at the level of the entire
positive integer space.

\subsection{Strategy of the Proof}

This paper adopts a structural approach based on decomposing the
positive integers into arithmetic residue classes determined by parity
patterns under the accelerated Collatz map.

The central idea is to study finite Collatz paths abstractly, independent
of their starting values, and to associate each such path with a
non-overlapping residue class of integers that follow the same parity
pattern for a fixed number of odd steps. Each class is assigned a
minimal stopping time, defined as the first iterate at which its
elements strictly decrease.

The proof proceeds through the following steps:
\begin{itemize}
\item Encode Collatz trajectories using parity words that record the
      number of divisions by two between odd steps.
\item Show that each admissible parity word induces an arithmetic
      residue class of positive integers.
\item Prove that convergent residue classes of fixed stopping time are
      pairwise disjoint.
\item Compute the asymptotic density of each class using its associated
      total division count.
\item Sum the densities of all convergent classes up to a given stopping
      time and show that the complement of this union has density
      tending to zero.
\item Conclude that every positive integer eventually reaches a smaller
      value, establishing global descent and ruling out non-trivial
      cycles.
\end{itemize}

By focusing on exhaustion of the positive integers via convergent
structures rather than on individual trajectories, this approach
provides a global mechanism for descent under the Collatz map.


\section{Definitions and Preliminaries}

\subsection{The Accelerated Collatz Map}

\begin{define}
The \emph{accelerated Collatz map} is the function
\( f : \setN \rightarrow \setN \) defined by
\[
f(x) = \begin{cases}
\displaystyle
    \phantom{3x} \frac{x}{2} & \text {if } x \text{ is even}, \\[8pt]  % Insert some spacing between rows
\displaystyle
    \frac{3x+1}{2}           & \text {if } x \text{ is odd}. \\
\end{cases}
\]
This map combines each odd step of the classical Collatz iteration
\(x \mapsto 3x+1\) with the immediately following division by two.
\end{define}

\subsection{Stopping Time}

\begin{define}
The stopping time \( \sigma(x) \) of a positive integer \(x\) is the minimal integer \(k \in \setN\) such that
\[
f^{k}(x) < x.
\]
\end{define}

\subsection{Convergent Subsequences}

\begin{define}
A \emph{convergent subsequence of length \(k\)} is a finite sequence
\[
x, f(x), f^2(x), \ldots, f^k(x)
\]
such that \( f^k(x) < x \) and \(k\) is minimal.
\end{define}

\subsection{Residue Classes}

\begin{define}
Two positive integers \(x,y \in \setN\) are said to be in the same
\emph{convergent residue class of stopping time \(k\)} if their first
\(k\) Collatz iterates are identical.
\end{define}

\begin{define}[Convergent Residue Class]
Fix \(k \in \setN\) and \(x \in \setN\). The \emph{convergent residue class of stopping time \(k\)}
containing \(x\) is
\[
[x]_k := \{ y \in \setN \mid f^i(y) = f^i(x) \text{ for } i = 0, 1, \dots, k-1 \},
\]
where \(f\) is the accelerated Collatz map.  \([x]_k\) collects all positive integers whose
first \(k\) Collatz iterates are identical to those of \(x\).
\end{define}


\section{Division Counts and Path Encoding}

The structure of convergent residue classes is governed by the total number
of divisions by two incurred along a Collatz path. This quantity depends
only on the parity pattern of the path and not on the specific starting
integer.

\subsection{Total Division Count}

Let a Collatz path contain exactly \(k\) odd steps (``up-legs''). Between
successive odd steps, the accelerated Collatz map applies a finite number
of divisions by two. The cumulative number of such divisions determines
the arithmetic structure of the corresponding residue class.

\begin{define}[Total Division Count]
Let \(k \ge 1\). Define \(A020914(k)\) to be the total number of divisions
by two applied along an accelerated Collatz path with exactly \(k\)
odd steps.
\end{define}

This quantity coincides with the number of binary digits of \(3^k\),
as tabulated by \OEIS{A020914}, but its role here is purely arithmetic:
it records the total exponent of 2 accumulated in the denominator of the
iterated map.

\subsection{Residue Class Modulus}

Fix a parity pattern with exactly \(k\) odd steps and total division count
\(A020914(k)\). All integers realizing this parity pattern form an
arithmetic progression with common difference \(2^{A020914(k)}\).

\begin{lemma}
Let \(x,y \in \setN\) follow the same Collatz parity pattern of length \(k\).
Then
\[
x \equiv y \pmod{2^{A020914(k)}}.
\]
\end{lemma}

\begin{proof}
Each odd step contributes a factor of 3 in the numerator, while each even
step contributes a factor of 2 in the denominator. After \(k\) odd steps,
the accumulated denominator is exactly \(2^{A020914(k)}\), independent of
the starting integer, which fixes the residue class modulo this power of two.
\end{proof}
\subsection{Density of Convergent Classes}

\begin{proposition}
Each convergent residue class with \(k\) odd steps has asymptotic density
\[
2^{-A020914(k)}.
\]
\end{proposition}

\begin{proof}
Each class is an arithmetic progression with step size \(2^{A020914(k)}\)
and exactly one representative per period. The result follows.
\end{proof}

\subsection{Example}

For convergent subsequences with \(k=4\) odd steps, \(A020914(4)=7\).
Thus all associated residue classes consist of integers of the form
\[
x = 2^7 n + r, \quad n \in \setN,
\]
where the residue \(r\) depends on the specific parity pattern.
The spacing between consecutive elements is therefore \(2^7 = 128\).

The preceding example illustrates that convergent behavior depends only
on parity structure and accumulated division count, motivating an abstract
description of Collatz trajectories independent of their starting values.


\section{Grammar of Collatz Parity Words}

\subsection{Parity Words}

\begin{remark}[Parity Grammar as a Dynamical Encoding]
The parity-word grammar introduced in this section provides a symbolic
encoding of accelerated Collatz trajectories. This encoding is not
injective: many integers correspond to the same parity word, forming
arithmetic residue classes. However, the grammar preserves the key
dynamical feature relevant to convergence, namely eventual descent.

Each parity word induces an affine transformation whose linear
coefficient determines whether all sufficiently large integers in the
associated residue class decrease after a fixed number of odd steps.
Thus, while individual trajectories are collapsed under the encoding,
their asymptotic behavior with respect to growth and descent is retained.
This suffices for density, exhaustion, and cycle-exclusion arguments.
\end{remark}

\begin{define}[Parity Word]
A \emph{parity word} of length \(k \ge 1\) is a finite sequence
\[
\omega = (d_1, d_2, \dots, d_k),
\]
where each \(d_i \in \setN\) denotes the number of divisions by two applied
after the \(i\)-th odd step of an accelerated Collatz trajectory.
\end{define}

As shown below, each parity word determines a linear-affine transformation
of the form
\(
x \mapsto \frac{3^k x + c(\omega)}{2^{d_1+\cdots+d_k}}.
\)

\begin{example}
Consider the parity word
\[
\omega = (1,1,1,4),
\]
which has length \(k=4\) and total division count
\[
d_1 + d_2 + d_3 + d_4 = 7 = A020914(4).
\]

This corresponds to a Collatz trajectory in which, after each of the first three
odd steps, the result is divided by two once, and after the fourth odd step,
it is divided by two four times.

The associated affine map is
\[
T_\omega(x) = \frac{3^4 x + c(\omega)}{2^7} = \frac{81x + c(\omega)}{128}.
\]

All positive integers realizing this parity word lie in a single arithmetic
residue class modulo \(2^7\) and follow the same parity pattern for four odd
steps. Since
\[
\frac{81}{128} < 1,
\]
this parity word is convergent, and all sufficiently large elements of the
associated residue class reach a strictly smaller integer after four odd steps.
\end{example}

\begin{example}[Concrete realization of $\omega=(1,1,1,4)$]
To illustrate how integrality arises for a parity word, consider the
positive integers realizing this pattern.

The first two positive integers realizing this parity word are
\[
x=15 \quad \text{and} \quad x=143,
\]
which satisfy
\[
15 \equiv 143 \pmod{2^7}.
\]

Evaluating the associated affine map on $x=15$ gives
\[
T_\omega(15) = \frac{81\cdot 15 + c(\omega)}{128}.
\]
Requiring $T_\omega(15)\in\setN$ yields $c(\omega)=65$, and hence
\[
T_\omega(15)=10.
\]

Using the same constant term,
\[
T_\omega(143)=\frac{81\cdot 143 + 65}{128}=91,
\]
confirming that all elements of the induced residue class follow the
same affine transformation.
\end{example}

\begin{lemma}[Associated Collatz Map]
Each parity word \(\omega = (d_1,\dots,d_k)\) induces an affine map
\[
T_\omega(x) = \frac{3^k x + c(\omega)}{2^{d_1+\cdots+d_k}},
\]
where \(c(\omega)\) is a nonnegative integer depending only on \(\omega\).
\end{lemma}

\begin{proof}
Each odd step contributes a factor of \(3\) to the numerator, while each
division by two contributes a factor of \(2\) to the denominator. The
constant term accumulates from the repeated application of the
\(3x+1\) operation and depends only on the parity pattern.
\end{proof}

\begin{remark}[Parity Words Encode Structure, Not Values]
Fixing a parity word $\omega$ determines the sequence of arithmetic
operations applied along an accelerated Collatz trajectory. Consequently,
both the linear coefficient $3^k/2^{d_1+\cdots+d_k}$ and the affine term
$c(\omega)$ depend only on $\omega$ and not on the starting integer.

Thus, convergence or divergence is a property of the parity word itself,
and applies uniformly to all sufficiently large elements of the induced
residue class.
\end{remark}

\begin{define}[Admissible Parity Word]
A parity word \(\omega\) is \emph{admissible} if there exists a positive
integer \(x\) such that \(T_\omega(x) \in \setN\).
\end{define}

\begin{lemma}[Minimal Division Count]
For any admissible parity word of length \(k\),
\[
d_1 + \cdots + d_k \ge A020914(k),
\]
where \(A020914(k)\) is the minimal power of two required to clear denominators in
an accelerated Collatz path with \(k\) odd steps.
\end{lemma}

\begin{proof}
Integrality of \(T_\omega(x)\) requires the denominator
\(2^{d_1+\cdots+d_k}\) to divide \(3^k x + c(\omega)\).
The minimal such exponent depends only on \(k\) and coincides with
\(A020914(k)\). The minimal such exponent is independent of \(x\).
\end{proof}

\begin{define}[Convergent Parity Word]
An admissible parity word \(\omega = (d_1,\dots,d_k)\) is
\emph{convergent} if
\[
\frac{3^k}{2^{d_1+\cdots+d_k}} < 1.
\]
\end{define}

\begin{remark}[Affine Term and Eventual Descent]
\label{rem:affine_term}
The accelerated Collatz map associated with a parity word
\(\omega = (d_1,\dots,d_k)\) is affine:
\[
T_\omega(x) = \frac{3^k}{2^{d_1+\cdots+d_k}}\,x
              + \frac{c(\omega)}{2^{d_1+\cdots+d_k}}.
\]
While the constant term \(c(\omega)\) arises from repeated applications
of the \(3x+1\) operation, convergence is determined by the linear
coefficient alone.  

If
\[
\frac{3^k}{2^{d_1+\cdots+d_k}} < 1,
\]
then there exists a finite threshold \(x_0\) such that
\(T_\omega(x) < x\) for all \(x > x_0\) within the induced residue class.
Thus, each convergent parity word guarantees eventual descent for all
but finitely many elements of its class. These finite exceptions do not
affect density, exhaustion, or cycle-exclusion arguments.
\end{remark}

\begin{lemma}
\label{lem:residue_class}
Each convergent parity word \(\omega\) induces a residue class
\[
x \equiv r_\omega \pmod{2^{A020914(k)}},
\]
whose elements all follow the same Collatz parity pattern for \(k\) odd steps and reach
an integer strictly smaller than the starting value after at most \(k\) odd steps.
\end{lemma}

\begin{proposition}
The number of convergent parity words of length \(k\) is given by
\OEIS{A186009}.
\end{proposition}


\section{Cumulative Density of Convergent Classes}

\subsection{Associative Grouping of Residue Classes}

Convergent residue classes may be grouped by stopping time without overlap.
That is, classes corresponding to distinct minimal stopping times form disjoint
subsets of \(\setN\), and each positive integer belongs to exactly one class of
minimal stopping time. This allows a cumulative accounting of the fraction of
integers accounted for by classes up to a given stopping time.

For convenience, we recall the definition of the accelerated Collatz map:
\[
f(x) = \begin{cases}
\displaystyle
    \phantom{3x} \frac{x}{2} & \text {if } x \text{ is even}, \\[8pt]  % Insert some spacing between rows
\displaystyle
    \frac{3x+1}{2}           & \text {if } x \text{ is odd}. \\
\end{cases}
\]

\begin{define}[Admissible Integer]
A positive integer \(x \in \setN\) is \emph{admissible of depth \(k\)} if
\[
\sigma(x) = k,
\]
where \(\sigma(x)\) is the minimal integer such that
\[
f^k(x) < x.
\]
Equivalently, \(k\) is the stopping time of \(x\).
\end{define}

\subsection{Lower Bound via Dominant Sequence}

\begin{theorem}[Cumulative Density Bound]
Let \(D_k\) denote the cumulative density of all convergent residue classes of stopping time
less than or equal to \(k\). Then \(D_k\) admits a strictly increasing sequence of lower bounds
\[
0 < D_1 < D_2 < \cdots < D_k < 1
\]
which converges to 1 as \(k \rightarrow \infty\).
\end{theorem}

\begin{proof}
Each convergent residue class of stopping time \(i\) has asymptotic density \(2^{-A020914(i)}\),
and the number of such classes is \(A186009(i)\).  For each \(k\), the cumulative density \(D_k\)
is obtained by summing the densities of all classes with stopping time \(i \le k\), yielding
a finite sum strictly less than 1 and strictly increasing in \(k\).

\[
D_k = \sum_{i \le k} A186009(i)\,2^{-A020914(i)}.
\]

The sequence \(A020914(i)\) satisfies
\[
A020914(i) \ge i \log_2 3,
\]
Since the number of convergent classes of stopping time $i$ is $A186009(i)$,
the tail contribution satisfies
\[
\sum_{i>k} A186009(i)\,2^{-A020914(i)}
\;\le\;
\sum_{i>k} C\,3^{-i}
\]
for some constant $C>0$, and hence converges to zero as $k\to\infty$.

Hence the complement of the union of convergent residue classes of stopping time \(\le k\) has
asymptotic density tending to zero, implying that \(D_k \to 1\).
\end{proof}

\subsection{Limit Argument}

\begin{corollary}[Exhaustion via Cumulative Density]
\label{cor:cum_density}
The cumulative density of convergent residue classes converges to 1:
\[
\lim_{k \to \infty} D_k = 1.
\]
Consequently, every positive integer belongs to some convergent residue class of finite stopping time.
\end{corollary}

\begin{proof}
This follows directly from the previous theorem. Since the complement of the
union of convergent residue classes of stopping time \(\le k\) has asymptotic
density tending to zero, every positive integer is eventually captured by some
convergent residue class.
\end{proof}


\section{Global Descent}

\begin{theorem}[Global Descent]
Every positive integer eventually reaches a smaller positive integer
under iteration of the Collatz map.
\end{theorem}

\begin{proof}
By Corollary~\ref{cor:cum_density}, the union of all convergent residue
classes has asymptotic density 1. Hence every positive integer belongs
to some convergent residue class of finite stopping time.

For each such class, there exists a finite integer \(k\) such that
\(f^k(x) < x\). Therefore, every positive integer eventually reaches
a smaller positive integer under iteration of the Collatz map.
\end{proof}


\section{Non-Existence of Non-Trivial Cycles}

\begin{theorem}
The only cycle in the positive integers under the Collatz map is
\(4 \rightarrow 2 \rightarrow 1 \rightarrow 4\).
\end{theorem}


\section{Conclusion}

We have shown that the positive integers may be partitioned into
non-overlapping residue classes determined by finite Collatz parity
patterns, each assigned a minimal stopping time. By explicitly
enumerating convergent parity words and summing the asymptotic densities
of their associated residue classes, we proved that the cumulative
density of convergent classes converges to one.

This establishes a global descent mechanism for the accelerated Collatz
map: every positive integer eventually reaches a strictly smaller value,
ruling out the existence of non-trivial cycles. The argument is global
and structural in nature, relying neither on probabilistic heuristics
nor on bounds for individual trajectories.

The approach reframes the Collatz conjecture as an exhaustion problem
over arithmetic classes rather than a question of trajectory control,
suggesting a framework that may be applicable to related discrete
dynamical systems.


\end{document}  
